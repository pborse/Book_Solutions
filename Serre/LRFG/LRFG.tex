\documentclass[oneside]{amsbook}
\usepackage{../../../book-solutions}

\setlist[enumerate]{label = (\alph*), leftmargin = *}
\renewcommand{\thepart}{\Roman{part}}
\renewcommand{\thesection}{\thechapter}

\title{Solutions to Jean-Pierre Serre's\\ \emph{Linear Representations of Finite Groups}}
\author{Patrick Borse}

\begin{document}
\begin{abstract}
This document contains solutions to the exercises of J-P. Serre's \emph{Linear Representations of Finite Groups}.
\end{abstract}

\maketitle

\tableofcontents

\part{Representations and Characters}
\setcounter{chapter}{1}
\chapter{Character theory}
\begin{prob}[Exercise on Sequences]

\end{prob}

\begin{prob}[Example: Sequences are Inadequate]

\end{prob}

\begin{prob}[Exercise on Hausdorff Spaces: Door Spaces]
Let $(X, \ms{T})$ be a Hausdorff door space, and let $s$ be an accumulation point of $X$. Then by Theorem 2.2(a), there is a net in $X\setminus\{s\}$ converging to $s$. Hence by Theorem 2.2(c), $X\setminus\{s\}$ is not closed and thus $\{s\}$ is not open. Since $X$ is a door space, it follows that $\{s\}$ is open and so $X\setminus\{s\}$ is closed. Now if $t \in X\setminus\{s\}$ is also an accumulation point of $X$, we have that $\{t\}$ is open and hence $X\setminus\{s, t\}$ is closed.
\end{prob}

\begin{prob}[Exercise on Subsequences]
If $m \in \omega$, then $\{i \in \omega\mid N_i < m\}$ is finite and so it has a maximal element $n$. Then if $p \in \omega$ with $p \geq n + 1$, we have $p \not \in \{i \in \omega\mid N_i < m\}$ and hence $N_p \geq m$. Thus for any sequence $S$, we have that $S\circ N$ is a subsequence of $S$.

Now suppose $N$ is a sequence of nonnegative integers such that $S\circ N$ is not a subsequence of $S$. Then by definition of a subsequence, there is $m \in \omega$ for which the set of $i \in \omega$ with $N_i \geq m$ is cofinal in $\omega$. By well-ordering of $\omega$, there is a least such $m$. Then by choice of $m$, $\{i \in \omega\mid N_i \leq m-1\}$ is not cofinal in $\omega$, and so it is bounded above. Thus $\{i \in \omega\mid N_i = m\}$ is cofinal in $\omega$, so also $\{i \in \omega\mid S_{N_i} = S_m\}$ is cofinal in $\omega$. Then $S_m$ is a cluster point of $S\circ N$.
\end{prob}

\begin{prob}[Example: Cofinal Subsets are Inadequate]
\begin{enumerate}
\item Since every point of $X$ other than $(0, 0)$ is open,it suffices to show that if $x \in X\setminus\{(0, 0)\}$ then $x$ and $(0, 0)$ can be separated by disjoint neighborhoods. But $\{x\}$ is open and also $X\setminus\{x\}$ is an open neighborhood of $(0, 0)$ since for all $m \in \omega$, $\{n \in \omega\mid (m, n) \in \{x\}\}$ has at most one element.

\item We showed in part (a) that if $x \in X\setminus\{(0, 0)\}$ then $\{x\}$ is closed and so the claim holds for $x$. On the other hand, $\{x\}$ is open for all $x \in X\setminus\{(0, 0)\}$ by definition of the topology and so $\{(0, 0)\}$ is the intersection of the closed neighborhoods $X\setminus\{x\}$ for $x \in X\setminus\{(0, 0)\}$. Since $X$ is countable by Theorem 0.17, we have by Theorem 0.15 that $X\setminus\{(0, 0)\}$ is countable and so $\{(0, 0)\}$ is a countable intersection of closed neighborhoods.

\item Let $\ms{U}$ be an open cover of $X$, and for each $x \in X$ let $U_x \in \ms{U}$ such that $x \in U_x$. Since $X$ is countable, $\{U_x\mid x \in X\}$ is countable by Theorem 0.16. Then $\{U_x\mid x \in X\}$ is a countable subcover of $\ms{U}$, so $X$ is Lindelöf.

\item 

\item 
\end{enumerate}
\end{prob}

\begin{prob}[Monotone Nets]
(Note: We assume that $>$ is antisymmetric, as in Problem 1.I.)
\begin{enumerate}
\item Let $\{S_n, n \in D, \succ\}$ be a monotone increasing net with bounded range, and let $s$ be the supremum of the range of $S$. Let $a, b \in X$ such that $a < s < b$. Then $a$ is not an upper bound of the range of $S$ since $<$ is antisymmetric. Thus there is $n \in D$ for which $a < S_n$. Then for $m \in D$ with $m \succ n$, $S_m \geq S_n$. Hence $a < S_m$. Since $s$ is an upper bound of the range of $S$, we have $a < S_m \leq s$ and so $a < S_m < b$ for $m \succ n$. Then $S$ is eventually in $\{x \in X\mid a < x < b\}$. But the collection of $\{x \in X\mid a < x < b\}$ for $a, b \in X$ is a base for the order topology on $X$ and so the collection of $\{x \in X\mid a < x < b\}$ for $a, b \in X$ with $a < s < b$ is a local base at $s$. Then $S$ converges to $s$.

\item 
\end{enumerate}
\end{prob}

\begin{prob}[Integration Theory, Junior Grade]
(Note: Since $\RR$ is Hausdorff, we have by Theorem 2.3 that nets in $\RR$ converge to at most one point. Thus if $f$ is summable over $A$, we can unambiguously write $\sum_Af$ for the unique point to which $S$ converges.)
\begin{enumerate}
\item Suppose that $f$ is nonnegative and that $\{S_F \mid F \in \ms{A}\}$ is bounded above. The net $S$ is monotone when $\RR$ is linearly ordered by $>$, for if $G\supset F$ with $F, G \in \ms{A}$ then $S_F \geq S_G$. Since $(\RR, >)$ is order-complete, we thus have by Problem 2.F(a) that $S$ converges to the supremum of $\{S_F\mid F \in \ms{A}\}$. Conversely, suppose for sake of contradiction that $f$ is nonnegative and summable but that $\{S_F\mid F \in \ms{A}\}$ is not bounded above. Let $S$ converge to $s \in \RR$. Let $a, b \in \RR$ such that $a < s < b$. There exists $F \in \ms{A}$ such that $S_F \geq b$, since $\{S_F\mid F \in \ms{A}\}$ is not bounded above. For any $G \in \ms{A}$ with $G \supset F$, we have since $f$ is nonnegative that $S_G \geq S_F$ and thus $S_G \geq b$. Since $(\ms{A}, \supset)$ is a directed set, it follows that $S$ is not eventually in $(a, b)$. But $(a, b)$ is a neighborhood of $s$, and so this contradicts the convergence of $S$ to $s$. Hence $\{S_F\mid F \in \ms{A}\}$ must be bounded above.

We now obtain the analogous result for nonpositive $f$ by replacing the linear order $>$ on $\RR$ with $<$. Let $f$ be nonpositive and suppose that $\{S_F\mid F \in \ms{A}\}$ is bounded below. We have that $S$ is a monotone net when $\RR$ is linearly ordered by $<$: if $F, G \in \ms{A}$ such that $G \supset F$, then $S_G \leq S_F$. Thus by Problem 2.F(a), $S$ converges to the infimum of $\{S_F\mid F \in \ms{A}\}$. Conversely, let $f$ be nonpositive and summable but suppose for sake of contradiction that $\{S_F\mid F \in \ms{A}\}$ is not bounded below. Let $S$ converge to $s \in \RR$. If $a, b \in \RR$ such that $a < s < b$, there exists $F \in \ms{A}$ such that $S_F \leq a$ since $\{S_F\mid F \in \ms{A}\}$ is not bounded below. Then for $G \in \ms{A}$ with $G \supset F$, we have $S_G \leq S_F$ so $S_G \leq a$. Now since $\ms{A}$ is directed by $\supset$, $S$ is not eventually in $(a, b)$, a contradiction. Hence $\{S_F\mid F \in \ms{A}\}$ is bounded below.

\item Let $\ms{A}_+$ denote the family of finite subsets of $A_+$ and $\ms{A}_-$ the family of finite subsets of $\ms{A}_-$. Then $\sum_Ff \geq 0$ for all $F \in \ms{A}_+$ and $\sum_Ff \leq 0$ for all $F \in \ms{A}_-$. Suppose first that $f$ is summable over $A$, and suppose for sake of contradiction that $\{S_{F_+}\mid F_+ \in \ms{A}_+\}$ is not bounded above. For $a \in \RR$ such that $\sum_Af < a$, there is $F \in \ms{A}$ such that $S_G < a$ for all $G \in \ms{A}$ with $G \supset F$. Since $F$ is finite, it has a finite number of subsets and hence there is $b \in \RR$ such that $b \leq S_{F'}$ for all $F' \subset F$. Then there is $F_+ \in \ms{A}_+$ such that $S_{F_+} \geq a - b$. We have that $F_+\cup F \in \ms{A}$ contains $F$, so $S_{F_+\cup F} < a$. Since \[S_{F_+\cup F} = S_{F_+} + S_{F\setminus F_+}\] and $F\setminus F_+ \subset F$, it follows that \[a = (a - b) + b \leq S_{F_+} + S_{F\setminus F_+} < a,\] a contradiction. Hence $\{S_{F_+}\mid F_+ \in \ms{A}_+\}$ is bounded above. If $\{S_{F_-}\mid F_- \in \ms{A}_-\}$ is not bounded below, let $a \in \RR$ such that $a < \sum_Af$ and let $b \in \RR$ such that $S_{F'} \leq b$ for all $F' \subset F$. Then there is $F \in \ms{A}$ such that $a < S_G$ for all $G \in \ms{A}$ with $G \supset F$, and $F_- \in \ms{A}_-$ such that $S_{F_-} \leq a-b$. Then $F_-\cup F \in \ms{A}$ contains $F$, so $a < S_{F_-\cup F}$. But \[S_{F_-\cup F} = S_{F_-} + S_{F\setminus F_-}\] and $F\setminus F_- \subset F$ implies \[a < S_{F_-} + S_{F\setminus F_-} \leq (a - b) + b = a.\] This is a contradiction, so $\{S_{F_-} \mid F_- \in \ms{A}_-\}$ is bounded below. By part (a), we thus have that $f$ is summable on $A_+$ and $A_-$.

Now suppose $f$ is summable over $A_+$ and $A_-$. For convenience, let $s_+ = \sum_{A_+}f$ and $s_- = \sum_{A_-}f$. Let $a, b \in \RR$ such that $a < s_+ + s_- < b$. We have \[\frac{a + s_+ - s_-}{2} =  s_+ + \frac{a - s_+ - s_-}{2}  < s_+ < s_+ + \frac{b - s_+ - s_-}{2} = \frac{b + s_+ - s_-}{2}\] and \[\frac{a - s_+ + s_-}{2} = s_- + \frac{a - s_+ - s_-}{2} < s_- < s_- + \frac{b - s_+ - s_-}{2} = \frac{b - s_+ + s_-}{2}.\] Then since $\{S_F, F \in \ms{A}_+, \supset\}$ converges to $s_+$ and $\{S_F, F \in \ms{A}_-, \supset\}$ converges to $s_-$, are $F_+ \in \ms{A}_+$ and $F_- \in \ms{A}_-$ such that for $G_+ \in \ms{A}_+$ with $G_+ \supset F_+$ and $G_- \in \ms{A}_-$ with $G_- \supset F_-$, \[\frac{a + s_+ - s_-}{2} < S_{G_+} < \frac{b + s_+ - s_-}{2}\] and \[\frac{a - s_+ + s_-}{2} < S_{G_-} < \frac{b - s_+ + s_-}{2}.\] Now let $F = F_+ \cup F_-$. Then for $G \in \ms{A}$ with $G \supset F$, let \[G_+ = \{a \in G\mid f(a) \geq 0\}\] and \[G_- = \{a \in G\mid f(a) < 0\}.\] Then $G_+ \subset A_+$ and $G_- \subset A_-$, and from $G_+, G_- \subset G$ we have that $G_+$ and $G_-$ are finite. Thus $G_+ \in \ms{A}_+$ and $G_- \in \ms{A}_-$. Moreover, $f$ is nonnegative on $F_+$ and $F_+ \subset F \subset G$, so $G_+ \supset F_+$. Similarly, $G_- \supset F_-$. Then \[\frac{a + s_+ - s_-}{2} < S_{G_+} < \frac{b + s_+ - s_-}{2}\] and \[\frac{a - s_+ + s_-}{2} < S_{G_-} < \frac{b - s_+ + s_-}{2}\] by the choice of $F_+$ and $F_-$. But $G$ is the disjoint union of $G_+$ and $G_-$, so $S_G = S_{G_+} + S_{G_-}$. Then from \[\frac{a + s_+ - s_-}{2} + \frac{a - s_+ + s_-}{2} = a\] and \[\frac{b + s_+ - s_-}{2} + \frac{b - s_+ + s_-}{2},\] we conclude that $a < S_G < b$. Hence $S$ is eventually in $(a, b)$, so $S$ converges to $s_+ + s_-$. That is, $f$ is summable over $A$ and \[\sum_A f = \sum_{A_+}f + \sum_{A_-}f.\]

In particular, if $f$ is summable over $A$, then $f$ is summable over $A_+$ and $A_-$ and so \[\sum_Af = \sum_{A_+}f + \sum_{A_-}f.\]

\item Suppose $f$ is summable over $A$. Then $f$ is summable over $A_+$ and $A_-$ by part (b). We have \[|f|(a) = \begin{cases}
f(a) & a \in A_+\\
-f(a) & a \in A_-
\end{cases}\] 

\item 

\item 

\item 

\item 

\item
\begin{enumerate}[label = (\roman*)]
\item 

\item 

\item 
\end{enumerate}
\end{enumerate}
\end{prob}

\begin{prob}[Integration Theory, Utility Grade]
\begin{enumerate}
\item 

\item 

\item 

\item 

\item 

\item 
\end{enumerate}
\end{prob}

\begin{prob}[Maximal Ideals in Lattices]
\begin{enumerate}
\item Let $\ms{A}$ denote the family of ideals which contain $A$ and are disjoint from $B$, and let $\ms{N}$ be a nest in $\ms{A}$. If $\ms{N}$ is empty, then since $A \in \ms{A}$, $A$ is an upper bound of $\ms{N}$ in $\ms{A}$. If $\ms{N}$ is nonempty, we claim that $\bigcup\ms{N} \in \ms{A}$. Clearly since $A \subset N$ for any $N \in \bigcup\ms{N}$, we have $A \subset \bigcup\ms{N}$. Moreover, \[\left(\bigcup\ms{N}\right)\cap B = \bigcup_{N \in \ms{N}}(N\cap B) = \bigcup_{N \in \ms{N}}\emptyset = \emptyset\] so $\bigcup\ms{N}$ is disjoint from $B$. Suppose $y \in \bigcup\ms{N}$ and $x \in X$ such that $y \geq x$. Then if $N \in \ms{N}$ such that $y \in N$, we have also that $x \in N$ since $N$ is an ideal of $X$. If $y, z \in \bigcup\ms{N}$, let $N, M \in \ms{N}$ such that $y \in N$ and $z \in M$. Then since $\ms{N}$ is a nest, WLOG $M \subset N$. Hence $y, z \in N$, and so $y\vee z \in N$. Thus $y\vee z \in \bigcup\ms{N}$, so that $\bigcup\ms{N}$ is an ideal of $X$. Now since $N \subset \bigcup\ms{N}$ for all $N \in \ms{N}$, we have that $\bigcup\ms{N}$ is an upper bound for $\ms{N}$ in $\ms{A}$. Hence by Theorem 0.25(a) (the Maximal Principle), there is a maximal element $A'$ of $\ms{A}$; that is, $A'$ is maximal among the ideals of $X$ containing $A$ and disjoint from $B$.

The existence of $B'$ is proven in exactly the same manner. Let $\ms{B}$ denote the family of dual ideals of $X$ which contain $B$ and are disjoint from $A'$. Since $B$ is a dual ideal of $X$ disjoint from $A'$, we have $B \in \ms{B}$. Thus the empty nest is bounded above in $\ms{B}$. Now suppose that $\ms{N}$ is a nonempty nest in $\ms{B}$; we wish to show that $\bigcup\ms{N} \in \ms{B}$. For any $N \in \ms{N}$, we have that $B \subset N$ and so $B \subset \bigcup\ms{N}$. Moreover, \[\left(\bigcup\ms{N}\right)\cap A' = \bigcup_{N \in \ms{N}}(N\cap A') = \bigcup\emptyset = \emptyset,\] so $\bigcup\ms{N}$ is disjoint from $A'$. Now let $y \in \bigcup\ms{N}$ and suppose $x \in X$ with $x\geq y$. Then if $N \in \ms{N}$ such that $y \in N$, we have $x \in N$ since $N$ is a dual ideal of $X$ and thus $x \in \bigcup\ms{N}$. For $y, z \in \bigcup\ms{N}$, there are $N, M \in \ms{N}$ such that $y \in N$ and $z \in M$. WLOG, since $\ms{N}$ is a nest, $M \subset N$. Then $y, z \in N$ and so since $N$ is a dual ideal, $y\wedge z \in N$. Thus $y \wedge z \in \bigcup\ms{N}$, and so $\bigcup\ms{N}$ is a dual ideal of $X$. Then $\bigcup\ms{N} \in \ms{B}$. Now by Theorem 0.25(a) (the Maximal Principle), there is a maximal element $B'$ of $\ms{B}$. This $B'$ is maximal among the dual ideals of $X$ containing $B$ and disjoint from $A'$.

\item For $c \in X$, let \[C = \{x \in X\mid x \leq c\text{ or } x \leq c\vee y\text{ for some }y \in A'\}.\] We claim that $C$ is an ideal of $X$ containing $A'$ and $c$. [TODO] If $C'$ is any ideal of $X$ containing $A'$ and $c$, then $c\vee y \in C'$ for any $y \in A'$. Then if $x \in X$ such that $x \leq c$ or $x \leq c\vee y$ for some $y \in A'$, we have $x \in C'$ since $C'$ is an ideal. Thus $C \subset C'$, and so $C$ is the smallest such ideal.

If $C$ is disjoint from $B$, then $C$ is an ideal of $X$ containing $A'$ (and thus also $A$) which is disjoint from $B$. Hence by maximality of $A'$, we have $C \subset A'$. In particular, $c \in A'$. Now suppose $c$ is in neither $A'$ nor $B$; then $C$ intersects $B$. Let $y \in B\cap C$. From $y \in C$, either $y \leq c$ or there exists $x \in A'$ such that $y \leq c\vee x$. But $B$ is a dual ideal and $c \not \in B$, so $y \leq c$ is impossible. Thus there is $x \in A'$ such that $y \leq c\vee x$, and so since $B$ is a dual ideal, $c\vee x \in B$.

\item Since $B \subset B'$, if $c$ is in neither $A'$ nor $B'$ then by part (b), there is $x \in A'$ such that $c\vee x \in B$. By an argument entirely analogous to part (b), there is $y \in B'$ such that $c\wedge y \in A'$. Then $(c\vee x)\wedge y \in B'$ since $B'$ is a dual ideal containing $c\vee x$ and $y$. We also have that \[(c\vee x)\wedge y = (c\wedge y)\vee (x\wedge y)\] since $X$ is distributive. From $x \geq x\wedge y$, we have $x\wedge y \in A'$ and thus $(c\wedge y)\vee (x\wedge y) \in A'$. But $B'$ is disjoint from $A'$ by construction, and so we have a contradiction. Then $A'\cup B' = X$, proving the claim.
\end{enumerate}
\end{prob}

\begin{prob}[Universal Nets]
\begin{enumerate}
\item Suppose $\{S_n, n \in D\}$ is a universal net in $X$ which is frequently in $A \subset X$. Then $S$ is not eventually in $X\setminus A$, and so $S$ is eventually in $A$ since it is a universal net.

\item We first show that if a net $\{S_n, n \in D\}$ in $X$ is eventually in $A \subset X$, then any subnet of $S$ is eventually in $A$. Indeed, let $N: E \to D$ be a function of directed sets as in the definition of a subnet of $S$. Since $S$ is eventually in $A$, there is $m \in D$ such that $S_n \in A$ for $n \geq m$. Let $n \in E$ such that for $p \geq n$, $N_p \geq m$. Then $S_{N_p} \in A$ for $p \geq n$, and so $S\circ N$ is eventually in $A$. Now if $S$ is a universal net in $X$, for any subset $A \subset X$, $S$ is eventually in $A$ or eventually in $X\setminus A$. Hence any subnet of $S$ is eventually in $A$ or eventually in $X\setminus A$, and so is a universal net in $X$.

Suppose $\{S_n, n \in D\}$ is a universal net in $X$ and that $f: X \to Y$ is a function. Then $f\circ S$ is a net in $Y$. For any subset $B \subset Y$, $S$ is eventually in $f^{-1}(B)$ or $X\setminus f^{-1}(B)$. Thus there is $n \in D$ such that either for all $m \geq n$, $S_n \in f^{-1}(B)$, or for all $m \geq n$, $S_n \in X\setminus f^{-1}(B)$. But $X\setminus f^{-1}(B) = f^{-1}(Y\setminus B)$ and hence for all $m \geq n$, $f(S_n) \in B$ or for all $m \geq n$, $f(S_n) \in Y\setminus B$. Hence $f\circ S$ is eventually in $B$ or $Y\setminus B$, so $f\circ S$ is a universal net in $Y$.

\item Let $S$ be a net in $X$ and let $\ms{A}$ be the family of all subsets $A \subset X$ for which $S$ is frequently in $A$, and let $\ms{P}$ be the family of subsets of $\ms{A}$ which are closed under finite intersections. Suppose that $\ms{N}$ is a nest in $\ms{P}$; we claim that $\bigcup\ms{N} \in \ms{P}$. Let $A, B \in \bigcup\ms{N}$. Then there are $N, M \in \ms{N}$ such that $A \in N$ and $B \in M$, and WLOG we assume that $M \subset N$. Then $A, B \in N$ and so $A\cap B \in N$. Hence $A\cap B \in \bigcup\ms{N}$, so $\bigcup\ms{N} \in \ms{P}$ (clearly $\bigcup\ms{N} \subset \ms{A}$). Thus $\ms{N}$ has an upper bound in $\ms{P}$, and so by Theorem 0.25(a) (the Maximal Principle), there is a maximal element of $\ms{P}$. Let $\ms{C}$ be a maximal element of $\ms{P}$. Suppose for sake of contradiction that there is $A \subset X$ such that $A$ and $X\setminus A$ are not in $\ms{C}$. To yield a contradiction, it suffices by definition of $\ms{C}$ to show that $\ms{C}\cup\{A\}$ or $\ms{C}\cup\{X\setminus A\}$ is in $\ms{P}$. [TODO: finish]

[TODO: other method]

\item Let $S$ be a net in $X$, and let $\ms{C}$ be as in part (c). By Lemma 2.5, there is a subnet $T$ of $S$ which is eventually in each member of $\ms{C}$. But for all $A \subset X$, we have $A \in \ms{C}$ or $X\setminus A \in \ms{C}$ and hence $T$ is eventually in $A$ or $X\setminus A$. Then $T$ is a universal subnet of $S$.
\end{enumerate}
\end{prob}

\begin{prob}[Boolean Rings: There are Enough Homomorphisms]
\begin{enumerate}
\item We have for $r, s \in R$ that
\begin{align*}
r + s & = (r + s)^2\\
& = r^2 + rs + sr + s^2\\
& = r + rs + sr + s,
\end{align*}
and so $0 = rs + sr$. Adding $rs$ to both sides yields $rs = sr$, and so $R$ is commutative.

\item Since $R$ is a ring, it has the usual $\ZZ$-algebra structure. For all $r \in R$, we have $2r = r + r = 0$ and thus this $\ZZ$-algebra structure factors through a $\ZZ/2\ZZ$-algebra structure.

\item If $A \in \ms{A}$, then
\begin{align*}
A\Delta\emptyset & = (A\cup\emptyset)\setminus(A\cap\emptyset)\\
& = A\setminus\emptyset\\
& = A.
\end{align*}
For $A, B \in \ms{A}$, it is clear that $A\Delta B = B\Delta A$. Moreover, we notice that
\begin{align*}
A\Delta B & = (A\cup B)\setminus(A\cap B)\\
& = (A\setminus(A\cap B))\cup(B\setminus(A\cap B))\\
& = (A\setminus B)\cup(B\setminus A).
\end{align*}
Now for $A, B, C \in \ms{A}$,
\begin{align*}
(A\Delta B)\Delta C & = ((A\Delta B)\setminus C)\cup (C\setminus(A\Delta B))\\
& = (((A\setminus B)\cup(B\setminus A))\setminus C)\cup(C\setminus((A\cup B)\setminus(A\cap B)))\\
& = (A\setminus(B\cup C))\cup(B\setminus(A\cup C))\cup(C\setminus(A\cup B))\cup(A\cap B\cap C)\\
& = (A\setminus(B\cup C))\cup(A\cap B\cap C)\cup(B\setminus(A\cup C))\cup(C\setminus(A\cup B))\\
& = (A\setminus((B\cup C)\setminus(B\cap C)))\cup(((B\setminus C)\setminus A)\cup((C\setminus B)\setminus A))\\
& = (A\setminus((B\cup C)\setminus(B\cap C)))\cup(((B\setminus C)\cup(C\setminus B))\setminus A)\\
& = (A\setminus(B\Delta C))\cup((B\Delta C)\setminus A)\\
& = A\Delta(B\Delta C).
\end{align*}
Thus $(\ms{A}, \Delta)$ is an abelian group. It is clear that $\cap$ is associative, and so we show that $\cap$ distributes over $\Delta$. For all $A, B, C \in \ms{A}$, we have
\begin{align*}
A\cap(B\Delta C) & = A\cap((B\cup C)\setminus(B\cap C))\\
& = (A\cap(B\cup C))\setminus(B\cap C)\\
& = ((A\cap B)\cup(A\cap C))\setminus(B\cap C)\\
& = ((A\cap B)\setminus(B\cap C))\cup((A\cap C)\setminus(B\cap C))\\
&  = ((A\cap B)\setminus(A\cap C))\cup((A\cap C)\setminus(A\cap B))\\
& = (A\cap B)\Delta(A\cap C),
\end{align*}
as desired. Moreover, for all $A \in \ms{A}$, we have $A\cap X = A$ and $X\cap A = A$ so $(\ms{A}, \Delta, \cap)$ is a ring with additive unit $\emptyset$ and multiplicative unit $X$. Finally, we have that $(\ms{A}, \Delta, \cap)$ is also a Boolean ring: for any $A \in \ms{A}$, it is clear that $A\cap A = A$ and \[A\Delta A = (A\cup A)\setminus(A\cap A) = A\setminus A = A.\]

\item [TODO]

\item Let $r, s, t \in R$ such that $r \geq s$ and $s\geq t$. Then \[r\cdot t = r\cdot(s\cdot t) = (r\cdot s)\cdot t = s\cdot t = t,\] so $r \geq t$. Thus $\geq$ partially orders $R$. For $r, s \in R$, let $r\vee s = r + s + r\cdot s$ and $r\wedge s = r\cdot s$. Then (note that by part (a), $R$ is commutative) $r\vee s = s\vee r$, $r\wedge s = s\wedge r$, and
\begin{align*}
(r\vee s)\cdot r & = (r + s + r\cdot s)\cdot r\\
& = r + s\cdot r + (r\cdot s)\cdot r\\
& = r + s\cdot r + (s\cdot r)\cdot r\\
& = r + s\cdot r + s\cdot r^2\\
& = r + s\cdot r + s\cdot r\\
& = r
\end{align*}
and
\begin{align*}
(r\wedge s)\cdot r & = (r\cdot s)\cdot r\\
& = r^2\cdot s\\
& = r\cdot s\\
& = r\wedge s.
\end{align*}
Then $r\vee s \geq r, s$ and $r, s \geq r\wedge s$. On the other hand, if $a \geq r, s$ and $r, s \geq b$, then
\begin{align*}
a\cdot(r\vee s) & = a\cdot(r + s + r\cdot s)\\
& = a\cdot r + a\cdot s + a\cdot(r\cdot s)\\
& = r + s + (a\cdot r)\cdot s\\
& = r + s + r\cdot s
\end{align*}
and
\begin{align*}
(r\wedge s)\cdot b & = (r\cdot s)\cdot b\\
& = r\cdot(s\cdot b)\\
& = r\cdot b\\
& = b.
\end{align*}
Thus $a \geq r\vee s$ and $r\wedge s \geq b$, so $r\vee s$ is the join of $r, s$ and $r\wedge s$ is the meet of $r, s$. For $r, s, t \in R$,
\begin{align*}
(r\vee s)\vee t & = (r + s + r\cdot s)\vee t\\
& = (r + s + r\cdot s) + t + (r + s + r\cdot s)\cdot t\\
& = r + s + r\cdot s + t + r\cdot t + s\cdot t + (r\cdot s)\cdot t\\
& = r + (s + t + s\cdot t) + r\cdot(s + t + s\cdot t)\\
& = r\vee(s + t + s\cdot t)\\
& = r\vee(s\vee t)
\end{align*}
and
\begin{align*}
(r\wedge s)\wedge t & = (r\cdot s)\cdot t\\
& = r\cdot(s\cdot t)\\
& = r\wedge(s\wedge t).
\end{align*}
Then $\vee$ and $\wedge$ are associative. Moreover, we have for $r, s, t \in R$ that
\begin{align*}
r\wedge(s\vee t) & = r\cdot(s + t + s\cdot t)\\
& = r\cdot s + r\cdot t + r\cdot(s\cdot t)\\
& = r\cdot s + r\cdot t + r^2\cdot(s\cdot t)\\
& = r\cdot s + r\cdot t + (r\cdot s)\cdot(r\cdot t)\\
& = (r\cdot s)\vee(r\cdot t)\\
& = (r\wedge s)\vee(r\wedge t)
\end{align*}
and
\begin{align*}
r\vee(s\wedge t) & = r\vee(s\cdot t)\\
& = r + s\cdot t + r\cdot(s\cdot t)\\
& = r^2 + r\cdot t + r\cdot t + s\cdot r + s\cdot t + r\cdot(s\cdot t) + r\cdot(s\cdot t) + r\cdot(s\cdot t)\\
& = r^2 + r\cdot t + r^2\cdot t + s\cdot r + s\cdot t + s\cdot(r\cdot t) + (r\cdot s)\cdot r + (r\cdot s)\cdot t + (r\cdot s)\cdot(r\cdot t)\\
& = (r + s + r\cdot s)\cdot(r + t + r\cdot t)\\
& = (r\vee s)\wedge(r\vee t)
\end{align*}
Thus $(R, \geq)$ is a distributive latttice.

\item [TODO]

\item [TODO]

\item [TODO]

\item [TODO]

\item [TODO]
\end{enumerate}
\end{prob}

\begin{prob}[Filters]
\begin{enumerate}
\item Suppose $U$ is an open subset of $X$, $x \in U$, and $\ms{F}$ is a filter in $X$ which converges to $x$. Then $U$ is a neighborhood of $x$, so $U \in \ms{F}$.

Conversely, suppose that $U \subset X$ such that $U$ belongs to every filter which converges to a point of $U$. By Theorem 1.2, for any $x \in U$ the neighborhood system $\ms{U}_x$ of $x$ is a dual ideal of the Boolean ring $(2^X, \Delta, \cap)$. Moreover, every neighborhood of $x$ is nonempty since it contains $x$, and hence $\ms{U}_x$ is a filter in $X$. It is immediate that $\ms{U}_x$ converges to $x$, and hence $U \in \ms{U}_x$. Then $U$ is a neighborhood of $x$, and so by Theorem 1.1, $U$ is open.

\item Suppose $A\setminus\{x\}$ belongs to some filter $\ms{F}$ in $X$ which converges to $x$. Then for all neighborhoods $U$ of $x$< we have $U \in \ms{F}$ and hence $U\cap(A\setminus\{x\}) \in \ms{F}$. Since $\emptyset\not\in\ms{F}$, it follows that $U$ intersects $A\setminus\{x\}$ and so $x$ is an accumulation point of $A$.

Conversely, suppose that $x$ is an accumulation point of $A$. As shown in part (a), the neighborhood system $\ms{U}_x$ of $x$ is a filter in $X$. By Problem 2.I(b), the smallest dual ideal of $(2^X, \Delta, \cap)$ containing $\ms{U}_x$ and $A\setminus\{x\}$ is \[\ms{F} = \{B \subset X\mid A\setminus\{x\} \subset B\text{ or } U\cap(A\setminus\{x\}) \subset B\text{ for some }U \in \ms{U}_x\}.\] But any neighborhood $U$ of $x$ intersects $A\setminus\{x\}$ since $x$ is an accumulation point of $A$, and thus $\ms{F}$ consists of nonempty sets. Hence $\ms{F}$ is a filter in $X$ which converges to $x$ and contains $A\setminus\{x\}$.

\item By definition, every $\ms{F} \in \phi_x$ contains $\ms{U}_x$. Thus $\ms{U}_x \subset \bigcap\phi_x$. Conversely, we saw in part (a) that $\ms{U}_x$ is a filter. Then $\ms{U}_x \in \phi_x$ and so $\bigcap\phi_x \subset \ms{U}_x$, proving the claim.

\item Since $\ms{F}$ converges to $x$, we have $\ms{U}_x \subset \ms{F}$. Then from $\ms{F} \subset \ms{G}$, we also have $\ms{U}_x \subset \ms{G}$ and so $\ms{G}$ converges to $x$.

\item Suppose for sake of contradiction that $A, B \subset X$ such that $A, B \not \in \ms{F}$ but $A\cup B \in \ms{F}$. Then since $\ms{F}$ is an ultrafilter, we must have that the smallest dual ideal of $(2^X, \Delta, \cap)$ containing $\ms{F}$ and $A$ is $2^X$ and the smallest dual ideal containing $\ms{F}$ and $B$ is $2^X$. Then by Problem 2.I(b), there exist $F, G \in \ms{F}$ such that $A\cap F = B\cap G = \emptyset$. Then $(A\cup B)\cap(F\cap G) \in \ms{F}$, but
\begin{align*}
(A\cup B)\cap(F\cap G) & = (A\cap(F\cap G))\cup(B\cap (F\cap G))\\
& = ((A\cap F)\cap G)\cup((B\cap G)\cap F)\\
& = \emptyset
\end{align*}
since $(A\cap F)\cap G \subset A\cap F = \emptyset$ and $(B\cap G)\cap F \subset B\cap G = \emptyset$. This is a contradiction (since $\emptyset\not\in\ms{F}$), so $A$ or $B$ is in $\ms{F}$.

If $X = \emptyset$, then the empty filter is the only ultrafilter in $X$ and so the second claim in the problem statement is false. Now suppose $X \not = \emptyset$. Then $\{X\}$ is a filter in $X$ which contains $\emptyset$ and so $\emptyset$ is not an ultrafilter in $X$. Thus if $\ms{F}$ is an ultrafilter in $X$, $\ms{F}$ is nonempty and so it follows that $X \in \ms{F}$ as $X$ contains every subset of $X$. Then for any $A \subset X$, we have $A\cup(X\setminus A) = X \in \ms{F}$ so that $A$ or $X\setminus A$ lies in $\ms{F}$.

\item
\begin{enumerate}
\item Since the order on $D$ is reflexive, $\{x_n, n \in D\}$ is not eventually in $\emptyset$. Thus $\emptyset \not \in \ms{F}$. If $A, B \in \ms{F}$, let $n, m \in D$ such that $x_p \in A$ for all $p \geq n$ and $x_p \in B$ for all $p \geq m$. Since $D$ is directed, there is $p \in D$ such that $p \geq n, m$. Then for $q \geq p$, we have $q \geq n, m$ so $x_q \in A\cap B$. Then $A\cap B \in \ms{F}$. Finally, suppose $A \in \ms{F}$ and that $B$ is a subset of $X$ containing $A$. Then there is $n \in D$ such that $x_m \in A$ for $m \geq n$, and so also $x_m \in B$ for $m \geq n$. Thus $B \in \ms{F}$, so that $\ms{F}$ is a filter in $X$.

\item (Note: We assume $\ms{F}$ is nonempty so that $D$ is nonempty.) We first show that $D$ is directed by $\geq$. Since $\ms{F}$ is nonempty, there is $F \in \ms{F}$. But $\ms{F}$ is a filter and so $F \not = \emptyset$. Thus there is $x \in F$, so $(x, F) \in D$; that is, $D$ is nonempty. If $(x, F), (y, G), (z, H) \in \ms{F}$ such that $(z, H) \geq (y, G)$ and $(y, G) \geq (x, F)$, then $H \subset G$ and $G \subset F$. Thus $(z, H) \geq (x, F)$, so $\geq$ is a partial order on $D$. For any $(x, F) \in D$, we clearly have $F \subset F$ and so $(x, F) \geq (x, F)$. Finally, let $(x, F), (y, G) \in D$. Then $F, G \in \ms{F}$ so $F\cap G \in \ms{F}$ with $F\cap G \subset F, G$. Letting $z \in F\cap G$ ($F\cap G$ is nonempty since $\ms{F}$ is a filter), we then have $(z, F\cap G) \in D$ with $(z, F\cap G) \geq (x, F), (y, G)$. Hence $D$ is directed by $\geq$.

Now we have that $\{f(x, F), (x, F) \in D\}$ is a net in $X$. Suppose $A \in \ms{F}$, and let $x \in A$. Then for any $(y, B) \in D$ such that $(y, B) \geq (x, A)$, we have $y \in B$ and $B \subset A$. Hence $f(y, B) = y \in A$, so that $\{f(x, F), (x, F) \in D\}$ is eventually in $A$. Conversely, let $A \subset X$ such that $\{f(x, F), (x, F) \in D\}$ is eventually in $A$. Then there is $(x, F) \in D$ such that $y \in A$ whenever $(y, G) \geq (x, F)$. Since $(y, F) \geq (x, F)$ for all $y \in F$, we thus have $y \in A$ for all $y \in F$. Hence $F \subset A$, and so $A \in \ms{F}$ since $F \in \ms{F}$.
\end{enumerate}
\end{enumerate}
\end{prob}

\chapter{Subgroups, products, induced representations}
\begin{ex}
Suppose $\{s_n\}_{n \in \NN}$ converges. Then by Theorem 3.11(a), $\{s_n\}_{n \in \NN}$ is Cauchy. Hence for any $\epsilon > 0$, there is $N \in \NN$ such that $|s_n-s_m| < \epsilon$ for $n, m \geq N$. Then \[||s_n|-|s_m|| \leq |s_n-s_m| < \epsilon\] for $n, m \geq N$, and so $\{|s_n|\}_{n \in \NN}$ is a Cauchy sequence. Thus by Theorem 3.11(c), $\{|s_n|\}_{n \in \NN}$ converges.

We provide also a direct proof which does not rely on Cauchy sequences. Suppose $\lim_{n\to\infty}s_n = s$. If $s > 0$, there is $N \in \NN$ such that $s_n > 0$ for all $n \geq N$. Hence $|s_n| = s_n$ for $n \geq N$, and so $\lim_{n\to\infty}|s_n| = s$. Similarly, if $s < 0$, there is $N \in \NN$ such that $s_n < 0$ for $n \geq N$. Then $|s_n| = -s_n$ for $n \geq N$, and so $\lim_{n\to\infty}|s_n| = -s$. Finally, suppose $s = 0$. Then for any $\epsilon > 0$, there is $N \in \NN$ such that $|s_n| < \epsilon$ for $n \geq N$. That is, $\lim_{n\to\infty}|s_n| = 0$.

The converse is false; for example, let $s_n = (-1)^n$ for all $n \in \NN$. Then $\{s_n\}_{n \in \NN}$ diverges but $\{|s_n|\}_{n \in \NN}$ is constant and thus converges.
\end{ex}

\begin{ex}
We observe that for all $n \in \NN$, \[\left(\sqrt{n^2 + n} + n\right)\left(\sqrt{n^2 + n} - n\right) = \left(\sqrt{n^2 + n}\right)^2 - n^2 = n\] and thus \[\sqrt{n^2 + n} - n = \frac{n}{\sqrt{n^2 + n} + n} = \frac{1}{\sqrt{1 + \frac{1}{n}} + 1}.\] But we have for all $n \in \NN$ that \[1 < \sqrt{1 + \frac{1}{n}} < 1 + \frac{1}{2n}\] with $\lim_{n\to\infty}1 = 1$ and $\lim_{n\to\infty}(1 + 1/(2n)) = 1$. Hence $\lim_{n\to\infty}\sqrt{1 + 1/n} = 1$. Then \[\lim_{n\to\infty}\left(\sqrt{1 + \frac{1}{n}} + 1\right) = 2\] by Theorem 3.3(b) with \[\sqrt{1 + \frac{1}{n}} + 1 > 0\] for all $n \in \NN$, so that \[\lim_{n\to\infty}\frac{1}{\sqrt{1 + \frac{1}{n}} + 1} = \frac{1}{2}\] by Theorem 3.3(d). Hence \[\lim_{n\to\infty}\left(\sqrt{n^2 + n} - n\right) = \frac{1}{2}.\]
\end{ex}

\begin{ex}
We first note that each $s_n$ is positive; this will be used implicitly throughout the solution. By Theorem 3.14, it suffices to show that $\{s_n\}_{n \in \NN}$ is monotonically increasing and bounded above. We first show by induction that $s_n < 2$ for all $n \in \NN$. For $n = 1$, this claim is true since \[s_1^2 = 2 < 4 = 2^2.\] Now suppose $s_n < 2$ for some $n \in \NN$. Then \[s_{n+1}^2 = 2 + \sqrt{s_n} < 2 + \sqrt{2} < 4,\] so $s_{n+1} < 2$ as desired.

Now we show by induction that $s_n < s_{n+1}$ for all $n \in \NN$. We have \[s_1^2 = 2 < 2 + \sqrt{s_1} = s_2^2,\] so $s_1 < s_2$. Now suppose $s_n < s_{n+1}$ for some $n \in \NN$. Then \[s_{n+1}^2 = 2 + \sqrt{s_n} < 2 + \sqrt{s_{n+1}} = s_{n+2}^2,\] so \[s_{n+1} < s_{n+2},\] which proves the claim.
\end{ex}

\begin{ex}
We prove by induction that for all $m \in \NN$, \[s_{2m-1} = 1 - \frac{1}{2^{m-1}}\] and \[s_{2m} = \frac{1}{2} - \frac{1}{2^m}.\] We have by definition that $s_1 = 0$ and \[s_2 = \frac{s_1}{2} = \frac{0}{2} = 0,\] so the claim holds for $m = 1$. Now suppose it holds for some $m \in \NN$; then
\begin{align*}
s_{2(m+1)-1} & = s_{2m+1}\\
& = \frac{1}{2} + s_{2m}\\
& = \frac{1}{2} + \left(\frac{1}{2} - \frac{1}{2^m}\right)\\
& = 1 - \frac{1}{2^m}\\
& = 1 - \frac{1}{2^{(m+1)-1}}
\end{align*}
and
\begin{align*}
s_{2(m+1)} & = s_{2m+2}\\
& = \frac{s_{2m+1}}{2}\\
& = \frac{1 - \frac{1}{2^m}}{2}\\
& = \frac{1}{2} - \frac{1}{2^{m+1}},
\end{align*}
proving the claim. Then since $\lim_{m\to\infty}1/2^{m-1} = 0$, we have by Theorem 3.3(b) that \[\lim_{m\to\infty}s_{2m-1} = \lim_{m\to\infty}\left(1 - \frac{1}{2^{m-1}}\right) = 1,\] so $\{s_n\}_{n \in \NN}$ has a subsequence convergening to 1. But also $s_n < 1$ for all $n \in \NN$, and so \[\limsup_{n\to\infty}s_n = 1.\] Moreover, from $\lim_{m\to\infty}1/2^m = 0$, we have by Theorem 3.3(b) that \[\lim_{m\to\infty}s_{2m} = \lim_{m\to\infty}\left(\frac{1}{2} - \frac{1}{2^m}\right) = \frac{1}{2}\] and hence $\{s_n\}_{n \in \NN}$ has a subsequence converging to $1/2$. Let $x < 1/2$. Then $1/2 - x > 0$, so there is $M \in \NN$ for which \[\frac{1}{2} - x > \frac{1}{2^M}.\] Then for any $m \in \NN$, we have \[\frac{1}{2} - x > \frac{1}{2^m}\] and so \[\frac{1}{2} - \frac{1}{2^m} > x.\] Thus $s_{2m} > x$, and so also $s_{2m-1} > x$ since \[s_{2m-1} = 2s_{2m} > s_{2m}\] as $s_{2m} > 0$. Then with $N = 2M - 1$, we have that $s_n > x$ for all $n \geq N$. Hence by Theorem 3.17, \[\liminf_{n\to\infty}s_n = \frac{1}{2}.\]
\end{ex}

\begin{ex}
Let $E_{a+b}$ be the set of all $x \in \bar{\RR}$ such that $a_{n_k} + b_{n_k} \to x$ for a subsequence $\{a_{n_k} + b_{n_k}\}_{k \in \NN}$ of $\{a_n + b_n\}_{n \in \NN}$; let $E_a$ and $E_b$ be defined similarly for the sequences $\{a_n\}_{n \in \NN}$ and $\{b_n\}_{n \in \NN}$. Then $E_{a+b} \subset E_a + E_b$, and so \[\sup E_{a + b} \leq \sup(E_a + E_b).\] If $\limsup_{n\to\infty}a_n$ and $\limsup_{n\to\infty}b_n$ are both real, then $\sup E_a + \sup E_b$ is an upper bound for $E_a + E_b$. Thus\[\sup(E_a + E_b) \leq \sup E_a + \sup E_b.\] Hence \[\sup E_{a+b} \leq \sup E_a + \sup E_b,\] that is, \[\limsup_{n\to\infty}(a_n+b_n) \leq \limsup_{n\to\infty}a_n + \limsup_{n\to\infty}b_n.\] If $\limsup_{n\to\infty}a_n$ or $\limsup_{n\to\infty}b_n$ is $\infty$, then there is nothing to prove. Finally, suppose WLOG that $\limsup_{n\to\infty}a_n = -\infty$ and $\limsup_{n\to\infty}b_n \not = \infty$. Then by Theorem 3.17(a), $E_a = \{-\infty\}$ and $\{b_n\}_{n \in \NN}$ is bounded above. Let $M \in \RR$ such that $b_n \leq M$ for all $n \in \NN$. By Theorem 3.6(b), it follows from $E_a = \{-\infty\}$ that every subsequence of $\{a_n\}_{n\to\infty}$ is unbounded below. Hence if $\{a_{n_k}+b_{n_k}\}_{k \in \NN}$ is any subsequence of $\{a_n+b_n\}_{n \in \NN}$, we see from $a_{n_k}+b_{n_k} \leq a_{n_k} + M$ that $\{a_{n_k}+b_{n_k}\}_{k \in \NN}$ is unbounded below. Thus $E_{a+b} = -\infty$, so again we have that $\sup E_{a+b} \leq \sup E_a+\sup E_b$. Then \[\limsup_{n\to\infty}(a_n + b_n) \leq \limsup_{n\to\infty}a_n + \limsup_{n\to\infty}b_n\] in all cases.
\end{ex}

\begin{ex}
\begin{enumerate}
\item It is clear that the $n$th partial sum of $\sum_{n = 1}^{\infty}a_n$ is $\sqrt{n+1}$. Since $\{\sqrt{n+1}\}_{n \in \NN}$ is unbounded above, we thus have by Theorem 3.2(c) that $\sum_{n = 1}^{\infty}a_n$ diverges.

\item For each $n \in \NN$, we observe that
\begin{align*}
|a_n| & = \frac{\sqrt{n+1} - \sqrt{n}}{n}\\
& = \frac{1}{n(\sqrt{n + 1} + \sqrt{n})}\\
& < \frac{1}{n\sqrt{n}}\\
& = \frac{1}{n^{3/2}}.
\end{align*}
By Theorem 3.28, $\sum_{n = 1}^{\infty}1/n^{3/2}$ converges since $3/2 > 1$. Thus by Theorem 3.25(a), $\sum_{n = 1}^{\infty}a_n$ converges.

\item We have for all $n \in \NN$ that
\begin{align*}
\sqrt[n]{|a_n|} & = \sqrt[n]{\left|(\sqrt[n]{n} - 1)^n\right|}\\
& = \sqrt[n]{n} - 1.
\end{align*}
But $\lim_{n\to\infty}(\sqrt[n]{n} - 1) = 0$ by Theorem 3.20(c) and Theorem 3.3(b). Thus by Theorem 3.33(a), $\sum_{n = 1}^{\infty}a_n$ converges.

\item If $|z| \leq 1$, then for any $n \in \NN$,
\begin{align*}
|a_n| & = \left|\frac{1}{1 + z^n}\right|\\
& = \frac{1}{|1 + z^n|}\\
& \geq \frac{1}{1 + |z|^n}\\
& \geq \frac{1}{2}.
\end{align*}
Thus $\{a_n\}_{n \in \NN}$ does not converge to 0, and so by Theorem 3.23, $\sum_{n = 1}^{\infty} a_n$ diverges.

Now suppose $|z| > 1$. Then for any $n \in \NN$,
\begin{align*}
\left|\frac{a_{n+1}}{a_n}\right| & = \left|\frac{\frac{1}{1 + z^{n+1}}}{\frac{1}{1 + z^n}}\right|\\
& = \frac{|1 + z^n|}{|1 + z^{n+1}|}\\
& \leq \frac{1 + |z|^n}{|z|^{n+1} - 1}\\
& = \frac{1}{|z|} + \frac{1 - \frac{1}{|z|}}{|z|^{n+1} - 1}.
\end{align*}
But since $|z| > 1$, we have that $|z|^{n+1} - 1 \to \infty$ as $n \to \infty$. Hence by Theorem 3.3(b), we have \[\lim_{n\to\infty}\left(\frac{1}{|z|} + \frac{1 - \frac{1}{|z|}}{|z|^{n+1} - 1}\right) = \frac{1}{|z|} < 1.\] Thus by Theorem 3.19, \[\limsup_{n\to\infty}\left|\frac{a_{n+1}}{a_n}\right| \leq \limsup_{n\to\infty}\left(\frac{1}{|z|} + \frac{1 - \frac{1}{|z|}}{|z|^{n+1} - 1}\right) < 1,\] and so by Theorem 3.34(a), $\sum_{n = 1}^{\infty}a_n$ converges.
\end{enumerate}
\end{ex}

\begin{ex}
We have for all $n \in \NN$ that \[\left(\sqrt{a_n} - \frac{1}{n}\right)^2 \geq 0\] and thus \[a_n - \frac{2\sqrt{a_n}}{n} + \frac{1}{n^2} \geq 0.\] Rearranging, \[\frac{\sqrt{a_n}}{n} \leq \frac{1}{2}\left(a_n + \frac{1}{n^2}\right).\] By Theorem 3.28, $\sum_{n = 1}^{\infty}1/n^2$ converges and thus by Theorem 3.47, \[\sum_{n = 1}^{\infty}\frac{1}{2}\left(a_n + \frac{1}{n^2}\right)\] converges. Now since $\sqrt{a_n}/n \geq 0$ for all $n \in \NN$, we conclude from Theorem 3.25(a) that $\sum_{n = 1}^{\infty}\sqrt{a_n}/n$ converges.
\end{ex}

\begin{ex}

\end{ex}

\begin{ex}
\begin{enumerate}
\item We have by Theorem 3.3 and Theorem 3.20(c) that
\begin{align*}
\lim_{n\to\infty}\sqrt[n]{|n^3|} & = \lim_{n\to\infty}(\sqrt[n]{n})^3\\
& = \left(\lim_{n\to\infty}\sqrt[n]{n}\right)^3\\
& = 1.
\end{align*}
Hence by Theorem 3.39, the radius of convergence of $\sum_{n = 0}^{\infty}n^3z^n$ is \[R = \frac{1}{\limsup_{n\to\infty}\sqrt[n]{|n^3|}} = 1.\]

\item We observe for all $z \in \CC$ and $n \in \NN$ that \[\left|\frac{\frac{2^{n+1}}{(n+1)!}z^{n+1}}{\frac{2^n}{n!}z^n}\right| = \frac{2|z|}{n+1}.\] By Theorem 3.3(b) and Theorem 3.20(a), \[\lim_{n\to\infty}\frac{2|z|}{n+1} = 2|z|\lim_{n\to\infty}\frac{1}{n+1} = 0.\] Then by Theorem 3.34(a), $\sum_{n=1}^{\infty}\frac{2^n}{n!}z^n$ converges for all $z \in \CC$. Thus the radius of convergence of $\sum_{n=1}^{\infty}\frac{2^n}{n!}z^n$ is $\infty$.

\item For all $n \in \NN$, we see by Theorem 3.3(b, c, d) and Theorem 3.20(c) that
\begin{align*}
\lim_{n\to\infty}\sqrt[n]{\left|\frac{2^n}{n^2}\right|} & = \lim_{n\to\infty}\frac{2}{(\sqrt[n]{n})^2}\\
& = 2\frac{1}{\left(\lim_{n\to\infty}\sqrt[n]{n}\right)^2}\\
& = 2.
\end{align*}
Hence by Theorem 3.39, the radius of convergence of $\sum_{n = 1}^{\infty}\frac{2^n}{n^2}z^n$ is \[R = \frac{1}{\limsup_{n\to\infty}\sqrt[n]{\left|\frac{2^n}{n^2}\right|}} = \frac{1}{2}.\]

\item This computation is almost identical to that of part (c). By Theorem 3.3(b, c) and Theorem 3.20(c), we have that
\begin{align*}
\lim_{n\to\infty}\sqrt[n]{\left|\frac{n^3}{3^n}\right|} & = \lim_{n\to\infty}\frac{(\sqrt[n]{n})^n}{3}\\
& = \frac{1}{3}\left(\lim_{n\to\infty}\sqrt[n]{n}\right)^3\\
& = \frac{1}{3}.
\end{align*}
Then by Theorem 3.39, the radius of convergence of $\sum\frac{n^3}{3^n}z^n$ is \[R = \frac{1}{\limsup_{n\to\infty}\sqrt[n]{\left|\frac{n^3}{3^n}\right|}} = 3.\]
\end{enumerate}
\end{ex}

\begin{ex}
By Theorem 3.39, we wish to show that $\limsup_{n\to\infty}\sqrt[n]{|a_n|} \geq 1$. Since infinitely many of the $a_n$ are distinct from zero, there is a subsequence $\{a_{n_k}\}_{k \in \NN}$ of $\{a_n\}_{n \in \NN}$ for which $\sqrt[n]{|a_{n_k}|} \geq 1$ for all $k \in \NN$. Then since every subsequence of $\{a_{n_k}\}_{k\in\NN}$ and by Theorem 3.19, \[\limsup_{n\to\infty}a_n \geq \limsup_{k\to\infty}a_{n_k} \geq 1.\]
\end{ex}

\begin{ex}
\begin{enumerate}
\item Suppose for sake of contradiction that $\sum_{n = 1}^{\infty}\frac{a_n}{1+a_n}$ converges. Then by Theorem 3.23, \[\lim_{n\to\infty}\frac{a_n}{1+a_n} = 0.\] Hence there is $N \in \NN$ such that \[\frac{a_n}{1+a_n} < \frac{1}{2}\] for all $n \geq N$. But then \[2a_n < 1 + a_n\] and so $a_n < 1$ for $n \geq N$. Hence \[|a_n| = a_n < \frac{2a_n}{1+a_n}\] for all $n \geq N$, so by Theorem 3.25(a), $\sum_{n=1}^{\infty}a_n$ converges. This is a contradiction and thus $\sum_{n=1}^{\infty}\frac{a_n}{1+a_n}$ diverges whenever $\sum_{n=1}^{\infty}a_n$ diverges.

\item Fix $N, k \in \NN$. Since $a_n > 0$ for all $n \in \NN$, we have $s_{N+m} < s_{N+k}$ for all $m = 1, \ldots, k$ and hence
\begin{align*}
\frac{a_{N+1}}{s_{N+1}} + \cdots + \frac{a_{N+k}}{s_{N+k}} & \geq \frac{a_{N+1}}{s_{N+k}} + \cdots + \frac{a_{N+k}}{s_{N+k}}\\
& = \frac{a_{N+1} + \cdots + a_{N+k}}{s_{N+k}}\\
& = \frac{s_{N+k}-s_N}{s_{N+k}}\\
& = 1 - \frac{s_N}{s_{N+k}}.
\end{align*}

We have by Theorem 3.24 that $s_n \to \infty$ as $n \to \infty$. Hence for fixed $N \in \NN$, since $s_n > 0$ for all $n \in \NN$, \[\lim_{k\to\infty}\left(1 - \frac{s_N}{s_{N+k}}\right) = 1.\] By the inequality established above, along with Theorem 3.19, it follows that \[\limsup_{k\to\infty}\sum_{n = N+1}^{N+k}\frac{a_n}{s_n} \geq \limsup_{k\to\infty}\left(1 - \frac{s_N}{s_{N+k}}\right) = 1.\] By Theorem 3.22, if $\sum_{n=1}^{\infty}\frac{a_n}{s_n}$ converges then there exists $N \in \NN$ such that \[\limsup_{k\to\infty}\sum_{n = N+1}^{N+k}\frac{a_n}{s_n} < 1.\] Hence $\sum_{n=1}^{\infty}\frac{a_n}{s_n}$ diverges.

\item From $a_n > 0$ for all $n \in \NN$, we have $s_{n-1} < s_n$ and thus
\begin{align*}
\frac{a_n}{s_n^2} & = \frac{s_n - s_{n-1}}{s_n^2}\\
& < \frac{s_n-s_{n-1}}{s_{n-1}s_n}\\
& = \frac{1}{s_{n-1}} - \frac{1}{s_n}
\end{align*}
for $n \geq 2$. The $n$th partial sum of $\sum_{n=2}^{\infty}\left(\frac{1}{s_{n-1}}-\frac{1}{s_n}\right)$ is \[\frac{1}{s_1} - \frac{1}{s_n} = \frac{1}{a_1} - \frac{1}{s_n}.\] But as explained in part (b), $s_n \to \infty$ as $n \to \infty$ and so \[\sum_{n = 2}^{\infty}\left(\frac{1}{s_{n-1}} - \frac{1}{s_n}\right) = \frac{1}{a_1}.\] Now by Theorem 3.25(a), since $\frac{a_n}{s_n^2} > 0$ for all $n \in \NN$, $\sum_{n = 1}^{\infty}\frac{a_n}{s_n^2}$ converges.

\item It is clear that if $a_n = 1$ for all $n \in \NN$ then $\lim_{n\to\infty}a_n = 1$, so by Theorem 3.23, $\sum_{n = 1}^{\infty}$ diverges. In this case, \[\frac{a_n}{1 + na_n} = \frac{1}{1 + n} \geq \frac{1}{2n}\] for all $n \in \NN$. But $\sum_{n = 1}^{\infty}\frac{1}{2n}$ diverges by Theorem 3.47 and Theorem 3.28, so by Theorem 3.25(a), \[\sum_{n = 1}^{\infty}\frac{a_n}{1 + na_n}\] diverges. On the other hand, let $S$ denote the set of positive square numbers and suppose \[a_n = \begin{cases}
1 & n\in S\\
0 & n\not\in S.
\end{cases}\] Then since $S$ is infinite, $\{a_n\}_{n \in \NN}$ does not converge to 0. Thus by Theorem 3.23, $\sum_{n = 1}^{\infty}a_n$ diverges. But \[\frac{a_n}{1 + na_n} = \begin{cases}
\frac{1}{1 + n} & n \in S\\
0 & n\not\in S.
\end{cases}\] Then by Theorem 3.28 and Theorem 3.24, \[\sum_{n = 1}^{\infty}\frac{a_n}{1 + na_n}\] converges since $\sum_{n = 1}^{\infty}1/n^2$ converges.

Now let $\{a_n\}_{n \in \NN}$ be any sequence of positive real numbers. For any $n \in \NN$, we have that \[\frac{a_n}{1 + n^2a_n} < \frac{a_n}{n^2a_n} = \frac{1}{n^2}.\] Thus by Theorem 3.25(a) and Theorem 3.28 (with $p = 2$), we have that \[\sum_{n = 1}^{\infty}\frac{a_n}{1 + n^2a_n}\] converges.
\end{enumerate}
\end{ex}

\begin{ex}
We first note that $\sum_{k = n}^{\infty}a_k$ converges for all $n \in \NN$ by Theorem 3.25(a), and so each $r_n$ is well-defined. Moreover, $\{r_n\}_{n \in \NN}$ is monotonically decreasing sequence of positive reals since $a_n > 0$ for all $n \in \NN$. Finally, by Theorem 3.22, $\lim_{n\to\infty}r_n = 0$.
\begin{enumerate}
\item We observe for $m < n$,
\begin{align*}
\frac{a_m}{r_m} + \cdots + \frac{a_n}{r_n} & > \frac{a_m}{r_m} + \cdots + \frac{a_n}{r_m}\\
& = \frac{a_m + \cdots + a_n}{r_m}\\
& = \frac{r_m-r_{n+1}}{r_m}\\
& > \frac{r_m-r_n}{r_m}\\
& = 1 - \frac{r_n}{r_m}.
\end{align*}
Since $\lim_{n\to\infty}r_n = 0$, we have for fixed $m \in \NN$ that \[\lim_{n\to\infty}\left(1 - \frac{r_n}{r_m}\right) = 0.\] Thus by Theorem 3.19, we have that
\begin{align*}
\limsup_{n\to\infty}\sum_{k = m}^n\frac{a_k}{r_k} \geq \limsup_{n\to\infty}\left(1 - \frac{r_n}{r_m}\right) = 1
\end{align*}
for all $m \in \NN$. But if $\sum_{n = 1}^{\infty}\frac{a_n}{r_n}$ converges, then by Theorem 3.22, there is $m \in \NN$ such that \[\limsup_{n\to\infty}\sum_{k = m}^n\frac{a_k}{r_k} < 1.\] Hence $\sum_{n = 1}^{\infty}\frac{a_n}{r_m}$ diverges.

\item We observe that for any $n \in \NN$, \[(\sqrt{r_n} + \sqrt{r_{n+1}})(\sqrt{r_n} - \sqrt{r_{n+1}}) = r_n - r_{n+1} = a_n.\] Thus
\begin{align*}
\sqrt{r_n} - \sqrt{r_{n+1}} & = \frac{a_n}{\sqrt{r_n} + \sqrt{r_{n+1}}}\\
& > \frac{a_n}{\sqrt{r_n} + \sqrt{r_n}}\\
& = \frac{a_n}{2\sqrt{r_n}}.
\end{align*}
Rearranging, \[\frac{a_n}{\sqrt{r_n}} < 2(\sqrt{r_n} - \sqrt{r_{n+1}})\] for all $n \in \NN$. Then the $n$th partial sum of $\sum_{n = 1}^{\infty}\frac{a_n}{\sqrt{r_n}}$ is bounded above by \[2\sqrt{r_1} = 2\sqrt{\sum_{k = 1}^{\infty}a_k}.\] By Theorem 3.24, it follows that $\sum_{n = 1}^{\infty}\frac{a_n}{\sqrt{r_n}}$ converges.
\end{enumerate}
\end{ex}

\begin{ex}
Let $\sum_{n = 0}^{\infty}a_n$ and $\sum_{n = 0}^{\infty}b_n$ be two absolutely convergent series, and let \[c_n = \sum_{k = 0}^na_kb_{n-k}\] for each $n \geq 0$. Then $\sum_{n = 0}^{\infty}|a_n|$ and $\sum_{n = 0}^{\infty}|b_n|$ converge absolutely, and so by Theorem 3.45 and Theorem 3.50, we have that \[\sum_{n = 0}^{\infty}\sum_{k = 0}^n|a_k||b_{n-k}|\] converges. But \[|c_n| = \left|\sum_{k = 0}^na_kb_{n-k}\right| \leq \sum_{k = 0}^n|a_k||b_{n-k}|.\] Thus by Theorem 3.25(a), $\sum_{n = 0}^{\infty}|c_n|$ converges, that is, the Cauchy product of $\sum_{n = 0}^{\infty}a_n$ and $\sum_{n = 0}^{\infty}b_n$ converges absolutely.
\end{ex}

\begin{ex}
\begin{enumerate}
\item Let $\epsilon > 0$. Then there is $N \in \NN$ such that $|s_n - s| < \epsilon$ for $n \geq N$. Then for all $n \geq N$, we have
\begin{align*}
|\sigma_n - s| & = \left|\frac{s_0 + s_1 + \cdots + s_n}{n + 1} - s\right|\\
& \leq \frac{|s_0 + s_1 + \cdots + s_{N-1}| + N|s| + |s_N + \cdots + s_n - (n-N+1)s|}{n+1}\\
& \leq \frac{|s_0 + s_1 + \cdots + s_{N-1}| + |s_N-s| + \cdots + |s_n-s|}{n+1}\\
& < \frac{|s_0+s_1+\cdots+s_{N-1}| + (n-N+1)\epsilon}{n+1}\\
& = \frac{|s_0+s_1+\cdots + s_{N-1}| - N\epsilon}{n+1} + \epsilon.
\end{align*}
Since $\lim_{n\to\infty}1/(n+1) = 0$, by Theorem 3.3(b) we have that \[\lim_{n\to\infty}\frac{|s_0+s_1+\cdots+s_{N-1}| - N\epsilon}{n+1} = 0.\] Hence by Theorem 3.19, \[\limsup_{n\to\infty}|\sigma_n-s| \leq \limsup_{n\to\infty}\left(\frac{|s_0 + s_1 + \cdots + s_{N-1}|-N\epsilon}{n+1} + \epsilon\right) = \epsilon.\] Since this holds for all $\epsilon > 0$, we have that $\limsup_{n\to\infty}|\sigma_n-s| = 0$ and hence $\lim_{n\to\infty}\sigma_n = s$ as desired.

\item Let $s_n = (-1)^n$ for all $n \geq 0$. Then \[\sigma_n = \begin{cases}
\frac{1}{n+1} & n\text{ is even}\\
0 & n\text{ is odd}.
\end{cases}\] Then since $\lim_{n\to\infty}1/(n+1) = 0$, we have that $\lim_{n\to\infty}\sigma_n = 0$. But clearly $\{s_n\}_{n\in\NN}$ does not converge since it is not Cauchy (Theorem 3.11(a)).

\item [TODO]

\item [TODO]

\item [TODO]
\end{enumerate}
\end{ex}

\begin{ex}

\end{ex}

\begin{ex}
\begin{enumerate}
\item It is clear by induction that $x_n > 0$ for all $n \in \NN$. We first show by induction that $x_n > \sqrt{\alpha}$ for all $n \in \NN$. The claim holds for $n = 1$ by assumption. If $x_n > \sqrt{\alpha}$ for some $n \in \NN$, then $x_n \not = \sqrt{\alpha}$ and hence \[\left(x_n - \frac{\alpha}{x_n}\right)^2 > 0.\] Thus \[4x_{n+1}^2 = \left(x_n + \frac{\alpha}{x_n}\right)^2 \geq 4\alpha,\] and so $x_{n+1} > \sqrt{\alpha}$, proving the claim.

Now for any $n \in \NN$, we have
\begin{align*}
x_{n+1} & = \frac{1}{2}\left(x_n + \frac{\alpha}{x_n}\right)\\
& < \frac{1}{2}\left(x_n + \frac{\alpha}{\sqrt{\alpha}}\right)\\
& = \frac{1}{2}(x_n + \sqrt{\alpha})\\
& < \frac{1}{2}(2x_n)\\
& = x_n.
\end{align*}
In particular, $\{x_n\}_{n\in\NN}$ is monotonically decreasing. [TODO: limit]

\item We observe for any $n \in \NN$ that
\begin{align*}
\frac{\epsilon_n^2}{2x_n} & = \frac{(x_n - \sqrt{\alpha})^2}{2x_n}\\
& = \frac{x_n^2-2x_n\sqrt{\alpha} + \alpha}{2x_n}\\
& = \frac{1}{2}\left(x_n + \frac{\alpha}{x_n}\right) - \sqrt{\alpha}\\
& = x_{n+1} - \sqrt{\alpha}\\
& = \epsilon_{n+1}.
\end{align*}
Thus since $x_n > \sqrt{\alpha}$ for all $n \in \NN$ (as shown in part (a)), we have \[\epsilon_{n+1} = \frac{\epsilon_n^2}{2x_n} < \frac{\epsilon_n^2}{2\sqrt{\alpha}}.\]

Now let $\beta = 2\sqrt{\alpha}$; we prove by induction that \[\epsilon_{n+1} < \beta\left(\frac{\epsilon_1}{\beta}\right)^{2^n}\] for all $n \in \NN$. For $n = 1$, we have by the above that
\begin{align*}
\epsilon_2 & < \frac{\epsilon_1^2}{2\sqrt{\alpha}}\\
& = 2\sqrt{\alpha}\left(\frac{\epsilon_1}{2\sqrt{\alpha}}\right)^{2^1}\\
& = \beta\left(\frac{\epsilon_1}{\beta}\right)^{2^1}.
\end{align*}
Now suppose for some $n \in \NN$ that \[\epsilon_{n+1} < \beta\left(\frac{\epsilon_1}{\beta}\right)^{2^n}.\] Then
\begin{align*}
\epsilon_{n+2} & < \frac{\epsilon_{n+1}^2}{2\sqrt{\alpha}}\\
& < \frac{1}{\beta}\left(\beta\left(\frac{\epsilon_1}{\beta}\right)^{2^n}\right)^2\\
& = \beta\left(\frac{\epsilon_1}{\beta}\right)^{2^{n+1}},
\end{align*}
proving the claim.

\item We observe that for $\alpha = 3$ and $x_1 = 2$, \[\frac{\epsilon_1}{\beta} = \frac{2 - \sqrt{3}}{2\sqrt{3}} = \frac{1}{\sqrt{3}} - \frac{1}{2}.\] Then since
\begin{align*}
\left(\frac{1}{\sqrt{3}}\right)^2 & = \frac{1}{3}\\
& < \frac{9}{25}\\
& = \left(\frac{3}{5}\right)^2\\
& = \left(\frac{1}{2} + \frac{1}{10}\right)^2,
\end{align*}
we have \[\frac{1}{\sqrt{3}} - \frac{1}{2} < \frac{1}{10}\] and so $\epsilon_1/\beta < 1/10$.

Since $\sqrt{3} < 2$ (as $3 < 4$), we have $\beta = 2\sqrt{3} < 4$. Thus by part (b),
\begin{align*}
\epsilon_{n+1} & < \beta\left(\frac{\epsilon_1}{\beta}\right)^{2^n}\\
& < 4\left(\frac{1}{10}\right)^{2^n}\\
& = 4\cdot 10^{-2^n}.
\end{align*}
For example, \[\epsilon_5 < 4\cdot 10^{-16}\] and \[\epsilon_6 < 4\cdot 10^{-32}.\]
\end{enumerate}
\end{ex}

\begin{ex}
\begin{enumerate}
\item We prove that $x_{2n-1} > x_{2n+1}$ for all $n \in \NN$. For any $n \in \NN$, we see that
\begin{align*}
x_{2n+1} & = x_{2n} + \frac{\alpha - x_{2n}^2}{1 + x_{2n}}\\
& = x_{2n-1} + \frac{\alpha - x_{2n-1}^2}{1 + x_{2n-1}} + \frac{\alpha - x_{2n}^2}{1 + x_{2n}}
\end{align*}

\item 

\item 

\item 
\end{enumerate}
\end{ex}

\begin{ex}
We assume that $\alpha$ is a positive real number and $x_1 > \sqrt[p]{\alpha}$. [TODO]
\end{ex}

\begin{ex}

\end{ex}

\begin{ex}
Suppose $\{p_{n_k}\}_{k \in \NN}$ is a subsequence of $\{p_n\}_{n \in \NN}$ which converges to $p$. Then for any $\epsilon > 0$, there is $K \in \NN$ such that $d(p_{n_k}, p) < \epsilon/2$ for all $k \geq K$. Since $\{p_n\}_{n \in \NN}$ is Cauchy, there also exists $N \in \NN$ such that $d(p_n, p_m) < \epsilon/2$ for $n, m \geq N$. For any $n \geq N$, we may choose $k \geq K$ such that $n_k \geq N$. Then
\begin{align*}
d(p_n, p) & \leq d(p_n, p_{n_k}) + d(p_{n_k}, p)\\
& < \epsilon/2 + \epsilon/2\\
& = \epsilon,
\end{align*}
and hence $\lim_{n\to\infty}p_n = p$.
\end{ex}

\begin{ex}
As in the proof of Theorem 3.10(b), we have from $E \subset E_n$ for all $n \in \NN$ and \[\lim_{n\to\infty}\diam E_n = 0\] that $\bigcap_{n = 1}^{\infty}E_n$ contains at most one point, so it suffices to show it is nonempty. For each $n \in \NN$, let $p_n \in E_n$ (since $E_n$ is nonempty). Then for each $N \in \NN$, we have $\{p_n\}_{n \geq N} \subset E_N$ and hence \[\lim_{N\to\infty}\diam\{p_n\}_{n \geq N} = 0\] since $\lim_{n\to\infty}\diam E_n = 0$. Thus $\{p_n\}_{n \in \NN}$ is Cauchy, and so it converges to some point $p$ since $X$ is complete. For any $N \in \NN$, the sequence $\{p_n\}_{n \geq N}$ in $E_N$ also converges to $p$. Hence $p$ is a limit point of $E_N$ for all $N \in \NN$, and so $p \in \bigcap_{n = 1}^{\infty}E_n$ since each $E_n$ is closed. This proves that $\bigcap_{n = 1}^{\infty}E_n$ is nonempty as desired.
\end{ex}

\begin{ex}

\end{ex}

\begin{ex}
As in the hint, we have for all $m, n \in \NN$ that \[d(p_n, q_n) \leq d(p_n, p_m) + d(p_m, q_m) + d(q_m, q_n).\] If $\epsilon > 0$, then there is $N \in \NN$ such that \[d(p_n, p_m) < \frac{\epsilon}{2}\] and \[d(q_n, q_m) < \frac{\epsilon}{2}\] for $n, m \geq N$. Then
\begin{align*}
d(p_n, q_n) - d(p_m, q_m) & \leq d(p_n, p_m) + d(q_m, q_n)\\
& < \frac{\epsilon}{2} + \frac{\epsilon}{2}\\
& = \epsilon,
\end{align*}
and so \[|d(p_n, q_n) - d(p_m, q_m)| < \epsilon\] (by interchanging $n$ and $m$) for $n, m \geq N$. Thus $\{d(p_n, q_n)\}_{n \in \NN}$ is a Cauchy sequence. By Theorem 3.11(c), it follows that $\{d(p_n, q_m)\}_{n \in \NN}$ converges.
\end{ex}

\begin{ex}
\begin{enumerate}
\item For any Cauchy sequence $\{p_n\}_{n \in \NN}$ in $X$, we have $d(p_n, p_n) = 0$ for all $n \in \NN$ and hence $\{p_n\}_{n \in \NN}$ is equivalent to $\{p_n\}_{n \in \NN}$. Suppose $\{p_n\}_{n \in \NN}$ and $\{q_n\}_{n \in \NN}$ are Cauchy sequences such that $\{p_n\}_{n \in \NN}$ is equivalent to $\{q_n\}_{n \in \NN}$. Then \[\lim_{n\to\infty}d(p_n, q_n) = 0,\] and so also \[\lim_{n\to\infty}d(q_n, p_n) = \lim_{n\to\infty}d(p_n, q_n) = 0.\] That is, $\{q_n\}_{n \in \NN}$ is equivalent to $\{p_n\}_{n \in \NN}$. Finally, suppose $\{p_n\}_{n \in \NN}$, $\{q_n\}_{n \in \NN}$, and $\{r_n\}_{n \in \NN}$ are Cauchy sequences in $X$ such that $\{p_n\}_{n \in \NN}$ is equivalent to $\{q_n\}_{n \in \NN}$ and $\{q_n\}_{n \in \NN}$ is equivalent to $\{r_n\}_{n \in \NN}$. Then for any $\epsilon > 0$, there is $N \in \NN$ such that \[d(p_n, q_n) < \frac{\epsilon}{2}\] and \[d(q_n, r_n) < \frac{\epsilon}{2}\] for $n\geq N$. Thus
\begin{align*}
d(p_n, r_n) & \leq d(p_n, q_n) + d(q_n, r_n)\\
& < \frac{\epsilon}{2} + \frac{\epsilon}{2}\\
& = \epsilon
\end{align*}
for $n \geq N$, and so $\lim_{n\to\infty}d(p_n, r_n) = 0$. Hence $\{p_n\}_{n \in \NN}$ is equivalent to $\{r_n\}_{n \in \NN}$, and so equivalence of Cauchy sequences in $X$ is an equivalence relation.

\item Let $P, Q \in X^*$ and supose $\{p_n\}_{n \in \NN}$ and $\{p_n'\}_{n \in \NN}$ are representatives of $P$ and $\{q_n\}_{n \in \NN}$ and $\{q_n'\}_{n \in \NN}$ are representatives of $Q$. Then we have for all $n \in \NN$ that
\begin{align*}
d(p_n, q_n) \leq d(p_n, p_n') + d(p_n', q_n') + d(q_n', q_n)
\end{align*}
so \[d(p_n, q_n) - d(p_n', q_n') \leq d(p_n, p_n') + d(q_n, q_n')\] and similarly \[d(p_n', q_n') - d(p_n, q_n) \leq d(p_n, p_n') + d(q_n, q_n').\] Hence \[|d(p_n, q_n) - d(p_n', q_n')| \leq d(p_n, p_n') + d(q_n, q_n').\] But $\lim_{n\to\infty}d(p_n, p_n') = 0$ and $\lim_{n\to\infty}d(q_n, q_n') = 0$ since $\{p_n\}_{n \in \NN}$ is equivalent to $\{p_n'\}_{n \in \NN}$ and $\{q_n\}_{n \in \NN}$ is equivalent to $\{q_n'\}_{n \in \NN}$. Thus \[\lim_{n\to\infty}(d(p_n, q_n) - d(p_n', q_n')) = 0,\] and so \[\lim_{n\to\infty}d(p_n, q_n) = \lim_{n\to\infty}d(p_n', q_n')\] by Theorem 3.3(a) and Exercise 3.23. Hence $\Delta(P, Q)$ is well-defined.

Now we show that $\Delta$ is a metric on $X^*$. Suppose $P, Q \in X^*$, and let $\{p_n\}_{n \in \NN}$ be a representative of $P$ and $\{q_n\}_{n \in \NN}$ a representative of $Q$. Then we have by Theorem 3.19 that \[\Delta(P, Q) = \lim_{n\to\infty}d(p_n, q_n) \geq 0\] since $d(p_n, q_n) \geq 0$ for all $n \in \NN$. Moreover, $\Delta(P, Q) = 0$ if and only if $\lim_{n\to\infty}d(p_n, q_n) = 0$, that is, $\{p_n\}_{n \in \NN}$ is equivalent to $\{q_n\}_{n \in \NN}$. Thus $\Delta(P, Q) = 0$ if and only if $P = Q$, so part (a) of Definition 2.15 is established. We also have that $d(p_n, q_n) = d(q_n, p_n)$ for all $n \in \NN$, and so \[\Delta(P, Q) = \lim_{n\to\infty}d(p_n, q_n) = \lim_{n\to\infty}d(q_n, p_n) = \Delta(Q, P).\] This proves part (b) of Definition 2.15. Finally, suppose also that $R \in X^*$ and $\{r_n\}_{n \in \NN}$ is a representative of $R$. Then we have \[d(p_n, r_n) \leq d(p_n, q_n) + d(q_n, r_n)\] for all $n \in \NN$ and thus by Theorem 3.19 and Theorem 3.3(a),
\begin{align*}
\Delta(P, R) & = \lim_{n\to\infty}d(p_n, r_n)\\
& \leq \lim_{n\to\infty}(d(p_n, q_n) + d(q_n, r_n))\\
& = \lim_{n\to\infty}d(p_n, q_n) + \lim_{n\to\infty}d(q_n, r_n)\\
& = \Delta(P, Q) + \Delta(Q, R).
\end{align*}
This is part (c) of Definition 2.15, and so $\Delta$ is a metric on $X^*$.

\item Let $\{P_k\}_{k \in \NN}$ be a Cauchy sequence in $X^*$. Let $\{p_{n, k}\}_{n \in \NN}$ be a representative of $P_k$ for each $k \in \NN$. For all $n, m \in \NN$, we have \[d(p_{n, n}, p_{m, m}) \leq d(p_{n, n}, p_{m, n}) + d(p_{m, n}, p_{m ,m}).\] For any $\epsilon > 0$, there is $K \in \NN$ such that \[\Delta(P_k, P_l) < \frac{\epsilon}{2}\] for $k, l \geq K$. 

\item We have that $\{p\}_{n \in \NN}$ is Cauchy since $d(p, p) = 0$; thus the class $P_p \in X^*$ is well-defined. By definition, for any $p, q \in X$, we have \[\Delta(P_p, P_q) = \lim_{n\to\infty}d(p, q) = d(p, q).\] That is, if $\phi: X \to X^*$ is given by $\phi(p) = P_p$, we have \[\Delta(\phi(p), \phi(q)) = d(p, q)\] for all $p, q \in X$. Then $\phi$ is an isometric embedding of $X$ into $X^*$ (note that by part (a) of Definition 2.15, a distance-preserving map of metric spaces is necessarily injective).

\item Let $P \in X^*$ and $\epsilon > 0$. Suppose $\{p_n\}_{n \in \NN}$ is a representative of $P$. Then $\{p_n\}_{n \in \NN}$ is Cauchy, and so there is $N \in \NN$ such that $d(p_n, p_m) < \epsilon/2$ for $n, m \geq N$. Thus \[\lim_{n\to\infty}d(p_n, p_N) \leq \frac{\epsilon}{2} < \epsilon\] by Theorem 3.19, and so \[\Delta(P, P_{p_N}) < \epsilon.\] But $P_{p_N} = \phi(p_N) \in \phi(X)$, and hence $\phi(X)$ is dense in $X^*$.

Now suppose $X$ is complete, and let $P \in X^*$. Let $\{p_n\}_{n \in \NN}$ be a representative of $P$. Then $\{p_n\}_{n \in \NN}$ is a Cauchy sequence in $X$, and so since $X$ is complete, there is $p \in X$ such that $\{p_n\}_{n \in \NN}$ converges to $p$. Thus \[\lim_{n\to\infty}d(p_n, p) = 0,\] and so $\{p_n\}_{n \in \NN}$ is equivalent to $\{p\}_{n \in \NN}$. Then $P = \phi(p)$, and so $\phi(X) = X^*$.
\end{enumerate}
\end{ex}

\begin{ex}

\end{ex}

\setcounter{chapter}{4}
\chapter{Examples}
\begin{prob}[Exercise on Real Functions on a Compact Space]
\begin{enumerate}
\item 

\item 

\item 
\end{enumerate}
\end{prob}

\begin{prob}[Compact Subsets]
\begin{enumerate}
\item 

\item 

\item 

\item 
\end{enumerate}
\end{prob}

\begin{prob}[Compactness Relative to the Order Topology]

\end{prob}

\begin{prob}[Isometries of Compact Metric Spaces]

\end{prob}

\begin{prob}[Countably Compact and Sequentially Compact Spaces]
\begin{enumerate}
\item 

\item 

\item 

\item 

\item 
\end{enumerate}
\end{prob}

\begin{prob}[Compactness; the Intersection of Comapct Connected Sets]
\begin{enumerate}
\item 

\item 
\end{enumerate}
\end{prob}

\begin{prob}[Problem on Local Compactness]

\end{prob}

\begin{prob}[Nest Characterization of Compactness]

\end{prob}

\begin{prob}[Complete Accumulation Points]

\end{prob}

\begin{prob}[Example: Unit Square With Dictionary Order]

\end{prob}

\begin{prob}[Example (the Ordinals) on Normality and Products]

\end{prob}

\begin{prob}[The Transfinite Line]

\end{prob}

\begin{prob}[Example: The Helly Space]
\begin{enumerate}
\item 

\item 

\item 

\item 
\end{enumerate}
\end{prob}

\begin{prob}[Exmaples on Closed Maps and Local Compactness]
\begin{enumerate}
\item 

\item 
\end{enumerate}
\end{prob}

\begin{prob}[Cantor Spaces]
\begin{enumerate}
\item 

\item 

\item 

\item 

\item 

\item 

\item 
\end{enumerate}
\end{prob}

\begin{prob}[Characterization of the Stone-Čech Compactification]

\end{prob}

\begin{prob}[Example (the Ordinals) on Compactification]

\end{prob}

\begin{prob}[The Wallman Compactification]
\begin{enumerate}
\item 

\item 

\item 

\item 

\item 

\item 

\item 
\end{enumerate}
\end{prob}

\begin{prob}[Boolean Rings: Stone Representation Theorem]
\begin{enumerate}
\item 

\item 

\item 

\item 

\item 
\end{enumerate}
\end{prob}

\begin{prob}[Compact Connected Spaces (the Chain Argument)]
\begin{enumerate}
\item 

\item 

\item 

\item 

\item 

\item 

\item 
\end{enumerate}
\end{prob}

\begin{prob}[Fully Normal Spaces]

\end{prob}

\begin{prob}[Point Finite Covers and Metacompact Spaces]
\begin{enumerate}
\item 

\item 

\item 
\end{enumerate}
\end{prob}

\begin{prob}[Partition of Unity]

\end{prob}

\begin{prob}[The Between Theorem for Semi-Continuous Functions]

\end{prob}

\begin{prob}[Paracompact Spaces]
\begin{enumerate}
\item 

\item 

\item 

\item 

\item 
\end{enumerate}
\end{prob}

\part{Representations in Characteristic Zero}
\chapter{The group algebra}
\begin{prob}[Exercise on Closed Relations]

\end{prob}

\begin{prob}[Exercise on the Product of Two Uniform Spaces]
\begin{enumerate}
\item 

\item 

\item 
\end{enumerate}
\end{prob}

\begin{prob}[A Discrete Non-Metrizable Uniform Space]

\end{prob}

\begin{prob}[Exercise: Uniform Spaces with a Nested Base]

\end{prob}

\begin{prob}[Example: A Very Incomplete Space (the Ordinals)]

\end{prob}

\begin{prob}[The Subbase Theorem for Total Boundedness]

\end{prob}

\begin{prob}[Some Extremal Uniformities]
\begin{enumerate}
\item 

\item 
\end{enumerate}
\end{prob}

\begin{prob}[Uniform Neighborhood Systems]
\begin{enumerate}
\item 

\item 

\item 
\end{enumerate}
\end{prob}

\begin{prob}[Écarts and Metrics]

\end{prob}

\begin{prob}[Uniform Covering Systems]

\end{prob}

\begin{prob}[Topologically Complete Spaces: Metrizable Spaces]
\begin{enumerate}
\item 

\item 

\item 
\end{enumerate}
\end{prob}

\begin{prob}[Topologically Complete Spcaes: Uniformizable Spaces]
\begin{enumerate}
\item 

\item 

\item 

\item 
\end{enumerate}
\end{prob}

\begin{prob}[The Discrete Subspace Argument; Countable Compactness]
\begin{enumerate}
\item 

\item 
\end{enumerate}
\end{prob}

\begin{prob}[Invariant Metrics]

\end{prob}

\begin{prob}[Topological Groups: Uniformities and Metrization]
\begin{enumerate}
\item 

\item 

\item 

\item 
\end{enumerate}
\end{prob}

\begin{prob}[Almost Open Subsets of a Topological Group]
\begin{enumerate}
\item 

\item 

\item 

\item 
\end{enumerate}
\end{prob}

\begin{prob}[Completion of Topological Groups]
\begin{enumerate}
\item 

\item 

\item 

\item 
\end{enumerate}
\end{prob}

\begin{prob}[Continuity and Openness of Homomorphisms: The Closed Graph Theorem]
\begin{enumerate}
\item 

\item 

\item 
\end{enumerate}
\end{prob}

\begin{prob}[Summability]
\begin{enumerate}
\item 

\item 

\item 
\end{enumerate}
\end{prob}

\begin{prob}[Uniformly Locally Compact Spaces]
\begin{enumerate}
\item 

\item 

\item 

\item 

\item 
\end{enumerate}
\end{prob}

\begin{prob}[The Uniform Boundedness Theorem]
\begin{enumerate}
\item 

\item 
\end{enumerate}
\end{prob}

\begin{prob}[Boolean $\sigma$-Rings]
\begin{enumerate}
\item 

\item 

\item 
\end{enumerate}
\end{prob}

\chapter{Induced representations; Mackey's criterion}
\begin{prob}[Exercise on the Topology of Pointwise Convergence]

\end{prob}

\begin{prob}[Exercise on Convergence of Functions]

\end{prob}

\begin{prob}[Pointwise Convergence on a Dense Subset]

\end{prob}

\begin{prob}[The Diagonal Process and Sequential Compactness]
\begin{enumerate}
\item 

\item 
\end{enumerate}
\end{prob}

\begin{prob}[Dini's Theorem]

\end{prob}

\begin{prob}[Continuity of an Induced Map]

\end{prob}

\begin{prob}[Uniform Equicontinuity]
\begin{enumerate}
\item 

\item 

\item 
\end{enumerate}
\end{prob}

\begin{prob}[Exercise on the Uniformity $\ms{U}\mid\ms{A}$]

\end{prob}

\begin{prob}[Continuity of Evaluation]

\end{prob}

\begin{prob}[Subspaces, Products, and Quotients of $k$-Spaces]
\begin{enumerate}
\item 

\item 

\item 
\end{enumerate}
\end{prob}

\begin{prob}[The $k$-Extension of a Topology]
\begin{enumerate}
\item 

\item 

\item 

\item 
\end{enumerate}
\end{prob}

\begin{prob}[Characterization of Even Continuity]

\end{prob}

\begin{prob}[Continuous Convergence]
\begin{enumerate}
\item 

\item 

\item 
\end{enumerate}
\end{prob}

\begin{prob}[The Adjoint of a Normed Linear Space]
\begin{enumerate}
\item 

\item 

\item 

\item 

\item 

\item 
\end{enumerate}
\end{prob}

\begin{prob}[Tietze Extension Theorem]

\end{prob}

\begin{prob}[Density Lemma for Linear Subspaces of $C(X)$]

\end{prob}

\begin{prob}[The Square Root Lemma for Banach Algebras]
\begin{enumerate}
\item 

\item 

\item 
\end{enumerate}
\end{prob}

\begin{prob}[The Stone-Weierstrass Theorem]
\begin{enumerate}
\item 
\begin{enumerate}
\item 

\item 

\item 
\end{enumerate}

\item 
\end{enumerate}
\end{prob}

\begin{prob}[Structure of $C(X)$]
\begin{enumerate}
\item 
\begin{enumerate}
\item 

\item 

\item 

\item 

\item 
\end{enumerate}

\item 

\item 

\item 

\item 

\item 

\item 
\end{enumerate}
\end{prob}

\begin{prob}[Compactification of Groups; Almost Periodic Functions]
\begin{enumerate}
\item 

\item 

\item 

\item 

\item 

\item 

\item 

\item 
\end{enumerate}
\end{prob}

\chapter{Examples of induced representations}
\begin{ex}

\end{ex}

\begin{ex}

\end{ex}

\begin{ex}

\end{ex}

\begin{ex}
\begin{enumerate}
\item 

\item 

\item 

\item 

\item 

\item 
\end{enumerate}
\end{ex}

\begin{ex}

\end{ex}

\begin{ex}
\begin{enumerate}
\item 

\item 

\item 
\end{enumerate}
\end{ex}

\begin{ex}

\end{ex}

\begin{ex}

\end{ex}

\begin{ex}

\end{ex}

\begin{ex}
\begin{enumerate}
\item 

\item 
\end{enumerate}
\end{ex}

\begin{ex}

\end{ex}

\begin{ex}
\begin{enumerate}
\item 

\item 
\end{enumerate}
\end{ex}

\begin{ex}
\begin{enumerate}
\item 

\item 
\end{enumerate}
\end{ex}

\begin{ex}

\end{ex}

\begin{ex}

\end{ex}

\begin{ex}
\begin{enumerate}
\item 

\item 

\item 
\end{enumerate}
\end{ex}

\begin{ex}
\begin{enumerate}
\item
\begin{enumerate}
\item 

\item 

\item 

\item 
\end{enumerate}

\item 
\end{enumerate}
\end{ex}

\chapter{Artin's theorem}
\input{Chapter_9}

\chapter{A theorem of Brauer}
\input{Chapter_10}

\chapter{Applications of Brauer's theorem}
\input{Chapter_11}

\chapter{Rationality questions}
\input{Chapter_12}

\chapter{Rationality questions: examples}
\input{Chapter_13}

\part{Introduction to Brauer Theory}
\chapter{The groups $R_K(G), R_k(G),$ and $P_k(G)$}
\input{Chapter_14}

\chapter{The $cde$ triangle}
\input{Chapter_15}

\chapter{Theorems}
\input{Chapter_16}

\chapter{Proofs}
\input{Chapter_17}

\chapter{Modular characters}
\input{Chapter_18}

\end{document}