\documentclass[oneside]{amsbook}
\usepackage{../../../book-solutions}

\setlist[enumerate]{label = (\alph*), leftmargin = *}
\renewcommand{\thepart}{\Roman{part}}
\renewcommand{\thesection}{\thechapter}

\title{Solutions to Jean-Pierre Serre's\\ \emph{Linear Representations of Finite Groups}}
\author{Patrick Borse}

\begin{document}
\begin{abstract}
This document contains solutions to the exercises of J-P. Serre's \emph{Linear Representations of Finite Groups}.
\end{abstract}

\maketitle

\tableofcontents

\part{Representations and Characters}
\setcounter{chapter}{1}
\chapter{Character theory}
\begin{prob}

\end{prob}

\begin{prob}

\end{prob}

\begin{prob}

\end{prob}

\begin{prob}

\end{prob}

\begin{prob}

\end{prob}

\begin{prob}

\end{prob}

\chapter{Subgroups, products, induced representations}
\begin{ex}

\end{ex}

\begin{ex}

\end{ex}

\begin{ex}

\end{ex}

\begin{ex}

\end{ex}

\begin{ex}

\end{ex}

\begin{ex}

\end{ex}

\begin{ex}

\end{ex}

\begin{ex}

\end{ex}

\begin{ex}

\end{ex}

\begin{ex}

\end{ex}

\begin{ex}

\end{ex}

\begin{ex}

\end{ex}

\begin{ex}

\end{ex}

\begin{ex}

\end{ex}

\begin{ex}

\end{ex}

\begin{ex}

\end{ex}

\begin{ex}

\end{ex}

\begin{ex}

\end{ex}

\begin{ex}

\end{ex}

\begin{ex}

\end{ex}

\setcounter{chapter}{4}
\chapter{Examples}
\begin{ex}

\end{ex}

\begin{ex}

\end{ex}

\begin{ex}

\end{ex}

\begin{ex}

\end{ex}

\begin{ex}

\end{ex}

\begin{ex}

\end{ex}

\begin{ex}

\end{ex}

\begin{ex}

\end{ex}

\begin{ex}

\end{ex}

\begin{ex}

\end{ex}

\begin{ex}

\end{ex}

\begin{ex}

\end{ex}

\begin{ex}

\end{ex}

\begin{ex}

\end{ex}

\begin{ex}

\end{ex}

\begin{ex}

\end{ex}

\begin{ex}

\end{ex}

\begin{ex}

\end{ex}

\begin{ex}

\end{ex}

\begin{ex}

\end{ex}

\begin{ex}

\end{ex}

\begin{ex}

\end{ex}

\part{Representations in Characteristic Zero}
\chapter{The group algebra}
\begin{ex}
\begin{enumerate}
\item 

\item 

\item 
\end{enumerate}
\end{ex}

\begin{ex}

\end{ex}

\begin{ex}

\end{ex}

\begin{ex}

\end{ex}

\begin{ex}

\end{ex}

\begin{ex}

\end{ex}

\begin{ex}

\end{ex}

\begin{ex}
\begin{enumerate}
\item 

\item 

\item 
\end{enumerate}
\end{ex}

\begin{ex}

\end{ex}

\begin{ex}

\end{ex}

\begin{ex}
\begin{enumerate}
\item 

\item 

\item 

\item 
\end{enumerate}
\end{ex}

\begin{ex}
\begin{enumerate}
\item 

\item 

\item 
\end{enumerate}
\end{ex}

\chapter{Induced representations; Mackey's criterion}
\begin{prob}

\end{prob}

\chapter{Examples of induced representations}
\begin{ex}

\end{ex}

\begin{ex}

\end{ex}

\begin{ex}

\end{ex}

\begin{ex}

\end{ex}

\begin{ex}

\end{ex}

\begin{ex}

\end{ex}

\begin{ex}

\end{ex}

\begin{ex}

\end{ex}

\begin{ex}

\end{ex}

\begin{ex}

\end{ex}

\begin{ex}

\end{ex}

\begin{ex}

\end{ex}

\begin{ex}

\end{ex}

\begin{ex}

\end{ex}

\begin{ex}

\end{ex}

\begin{ex}

\end{ex}

\begin{ex}

\end{ex}

\begin{ex}

\end{ex}

\begin{ex}

\end{ex}

\begin{ex}

\end{ex}

\chapter{Artin's theorem}
\begin{ex}
\begin{enumerate}
\item 

\item 
\end{enumerate}
\end{ex}

\begin{ex}

\end{ex}

\begin{ex}
\begin{enumerate}
\item 

\item 

\item 

\item 

\item 
\end{enumerate}
\end{ex}

\begin{ex}

\end{ex}

\begin{ex}
\begin{enumerate}
\item 

\item 

\item 

\item 
\end{enumerate}
\end{ex}

\begin{ex}
\begin{enumerate}
\item 

\item 

\item 

\item 
\end{enumerate}
\end{ex}

\begin{ex}

\end{ex}

\begin{ex}
\begin{enumerate}
\item 

\item 
\end{enumerate}
\end{ex}

\begin{ex}

\end{ex}

\chapter{A theorem of Brauer}
\begin{ex}

\end{ex}

\begin{ex}

\end{ex}

\begin{ex}

\end{ex}

\begin{ex}

\end{ex}

\begin{ex}

\end{ex}

\begin{ex}

\end{ex}

\begin{ex}

\end{ex}

\begin{ex}

\end{ex}

\begin{ex}

\end{ex}

\begin{ex}

\end{ex}

\begin{ex}

\end{ex}

\begin{ex}

\end{ex}

\begin{ex}

\end{ex}

\begin{ex}

\end{ex}

\begin{ex}

\end{ex}

\begin{ex}

\end{ex}

\begin{ex}

\end{ex}

\begin{ex}

\end{ex}

\begin{ex}

\end{ex}

\begin{ex}

\end{ex}

\begin{ex}

\end{ex}

\begin{ex}

\end{ex}

\begin{ex}

\end{ex}

\begin{ex}

\end{ex}

\begin{ex}

\end{ex}

\begin{ex}

\end{ex}

\begin{ex}

\end{ex}

\begin{ex}

\end{ex}

\begin{ex}

\end{ex}

\begin{ex}

\end{ex}

\chapter{Applications of Brauer's theorem}
\begin{ex}

\end{ex}

\begin{ex}

\end{ex}

\begin{ex}

\end{ex}

\begin{ex}

\end{ex}

\begin{ex}

\end{ex}

\begin{ex}

\end{ex}

\begin{ex}

\end{ex}

\begin{ex}

\end{ex}

\begin{ex}

\end{ex}

\begin{ex}

\end{ex}

\begin{ex}

\end{ex}

\begin{ex}

\end{ex}

\begin{ex}

\end{ex}

\begin{ex}

\end{ex}

\begin{ex}

\end{ex}

\begin{ex}

\end{ex}

\begin{ex}

\end{ex}

\begin{ex}

\end{ex}

\begin{ex}

\end{ex}

\begin{ex}

\end{ex}

\begin{ex}

\end{ex}

\begin{ex}

\end{ex}

\chapter{Rationality questions}
\begin{ex}

\end{ex}

\begin{ex}

\end{ex}

\begin{ex}

\end{ex}

\begin{ex}

\end{ex}

\begin{ex}

\end{ex}

\begin{ex}

\end{ex}

\begin{ex}

\end{ex}

\begin{ex}

\end{ex}

\begin{ex}

\end{ex}

\begin{ex}

\end{ex}

\begin{ex}

\end{ex}

\begin{ex}

\end{ex}

\begin{ex}

\end{ex}

\begin{ex}

\end{ex}

\begin{ex}

\end{ex}

\begin{ex}

\end{ex}

\begin{ex}

\end{ex}

\begin{ex}

\end{ex}

\begin{ex}

\end{ex}

\begin{ex}

\end{ex}

\begin{ex}

\end{ex}

\begin{ex}

\end{ex}

\begin{ex}

\end{ex}

\begin{ex}

\end{ex}

\begin{ex}

\end{ex}

\begin{ex}

\end{ex}

\begin{ex}

\end{ex}

\begin{ex}

\end{ex}

\begin{ex}

\end{ex}

\begin{ex}

\end{ex}

\begin{ex}

\end{ex}

\begin{ex}

\end{ex}

\begin{ex}

\end{ex}

\begin{ex}

\end{ex}

\chapter{Rationality questions: examples}
\begin{ex}

\end{ex}

\begin{ex}

\end{ex}

\begin{ex}

\end{ex}

\begin{ex}

\end{ex}

\begin{ex}

\end{ex}

\begin{ex}

\end{ex}

\begin{ex}

\end{ex}

\begin{ex}

\end{ex}

\begin{ex}

\end{ex}

\begin{ex}

\end{ex}

\begin{ex}

\end{ex}

\begin{ex}

\end{ex}

\part{Introduction to Brauer Theory}
\chapter{The groups $R_K(G), R_k(G),$ and $P_k(G)$}
\begin{prob}

\end{prob}

\chapter{The $cde$ triangle}
\begin{ex}

\end{ex}

\begin{ex}

\end{ex}

\begin{ex}

\end{ex}

\begin{ex}

\end{ex}

\begin{ex}

\end{ex}

\begin{ex}

\end{ex}

\begin{ex}

\end{ex}

\begin{ex}

\end{ex}

\begin{ex}

\end{ex}

\chapter{Theorems}
\begin{prob}

\end{prob}

\begin{prob}

\end{prob}

\begin{prob}

\end{prob}

\begin{prob}

\end{prob}

\begin{prob}

\end{prob}

\begin{prob}

\end{prob}

\chapter{Proofs}
\begin{ex}

\end{ex}

\begin{ex}

\end{ex}

\begin{ex}

\end{ex}

\begin{ex}

\end{ex}

\begin{ex}

\end{ex}

\begin{ex}

\end{ex}

\begin{ex}

\end{ex}

\begin{ex}

\end{ex}

\begin{ex}

\end{ex}

\begin{ex}

\end{ex}

\begin{ex}

\end{ex}

\begin{ex}

\end{ex}

\begin{ex}

\end{ex}

\begin{ex}

\end{ex}

\begin{ex}

\end{ex}

\begin{ex}

\end{ex}

\begin{ex}

\end{ex}

\begin{ex}

\end{ex}

\begin{ex}

\end{ex}

\begin{ex}

\end{ex}

\begin{ex}

\end{ex}

\begin{ex}

\end{ex}

\chapter{Modular characters}
\begin{prob}

\end{prob}

\end{document}