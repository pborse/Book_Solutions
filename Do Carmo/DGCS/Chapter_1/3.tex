\begin{ex}
The line in $\RR^3$ defined by $y = 0$ and $z = x$ is the $\RR$-span of $(1, 0, 1)$. Thus it suffices to show that $\alpha'(t)$ makes a constant angle with $(1, 0, 1)$. Indeed, we have \[\alpha'(t) = (3, 6t, 6t^2)\] for all $t \in \RR$ so if $\theta(t)$ is the angle between $\alpha'(t)$ and $(1, 0, 1)$ then $0\leq\theta(t)\leq\pi$ with
\begin{align*}
\cos\theta(t) & = \frac{\alpha'(t)\cdot (1, 0, 1)}{\|\alpha'(t)\|\|(1, 0, 1)\|}\\
& = \frac{3 + 6t^2}{\sqrt{3^2 + (6t)^2 + (6t^2)^2}\sqrt{2}}\\
& = \frac{3 + 6t^2}{3\sqrt{1 + 4t^2 + 4t^4}\sqrt{2}}\\
& = \frac{1 + 2t^2}{\sqrt{(1+2t^2)^2}\sqrt{2}}\\
& = \frac{1}{\sqrt{2}}
\end{align*}
since $1+2t^2 \geq 0$ for all $t \in \RR$. Then $\theta(t) = \pi/4$ is constant.
\end{ex}

\begin{ex}
\begin{enumerate}
\item As in the diagram, let $t$ be the angle swept out by the radius to the given point on the circumference. Then at time $t$, the center of the disk lies at $(t, 1)$ while the position of the point relative to the center is $(-\sin t, -\cos t)$. Hence $\alpha: \RR \to \RR^2$ given by \[\alpha(t) = (t - \sin t, 1 - \cos t)\] is a parameterized curve which traces the cycloid. For all $t \in \RR$, \[\alpha'(t) = (1 - \cos t, \sin t).\] Then $\alpha'(t) = 0$ if and only if $t \in 2\pi\ZZ$, so the singular points are $(2\pi n, 0)$ for $n \in \ZZ$.

\item By the choice of parameterization, a complete rotation of the disk occurs from $t = 0$ to $t = 2\pi$. The arc length of the cycloid on this interval is
\begin{align*}
\int_0^{2\pi}|\alpha'(t)|\,dt & = \int_0^{2\pi}\sqrt{(1-\cos t)^2 + \sin^2t}\,dt\\
& = \int_0^{2\pi}\sqrt{2-2\cos t}\,dt\\
& = \int_0^{2\pi}\sqrt{4\sin^2(t/2)}\,dt\\
& = \int_0^{2\pi}2\sin(t/2)\,dt\\
& = \int_0^{\pi}4\sin u\,du\\
& = (-4\cos u)\big|_0^{\pi}\\
& = 8.
\end{align*}
\end{enumerate}
\end{ex}

\begin{ex}
\begin{enumerate}
\item At any point on the circumference of a circle, the tangent line is perpendicular to the radius. Thus $AV$ is the line $x = 2a$. Then the ray $r$ making angle $\theta \in (-\pi/2, \pi/2)$ with the positive $x$-axis intersects $AV$ at $B = (2a, 2a\tan\theta)$ and intersects $S^1$ at $C = (t\cos\theta, t\sin\theta)$ for $t > 0$ such that \[(t\cos\theta - a)^2 + (t\sin\theta)^2 = a^2.\] Then \[t^2-2at\cos\theta = 0\] so $t = 2a\cos\theta$, and thus \[C = (2a\cos^2\theta, 2a\cos\theta\sin\theta).\] Then \[\bar{CB} = (2a, 2a\tan\theta) - (2a\cos^2\theta, 2a\cos\theta\sin\theta) = (2a\sin^2\theta, 2a\sin^2\theta\tan\theta),\] so \[p = (2a\sin^2\theta, 2a\sin^2\theta\tan\theta).\] We have that \[\frac{\tan^2\theta}{1 + \tan^2\theta} = \frac{\tan^2\theta}{\sec^2\theta} = \sin^2\theta,\] so \[p = \left(\frac{2a\tan^2\theta}{1 + \tan^2\theta}, \frac{2a\tan^2\theta}{1 + \tan^2\theta}\right).\] Thus since $\tan: (-\pi/2, \pi/2) \to \RR$ is a diffeomorphism, we see that the parameterized curve $\alpha: \RR \to \RR^2$ given by \[\alpha(t) = \left(\frac{2at^2}{1+t^2}, \frac{2at^3}{1+t^2}\right)\] traces the cissoid of Diocles.

\item Using the parameterized curve $\alpha: \RR \to \RR^2$ from part a., we have that
\begin{align*}
\alpha'(t) & = \left(\frac{(4at)(1+t^2) - (2at^2)(2t)}{(1+t^2)^2}, \frac{(6at^2)(1+t^2) - (2at^3)(2t)}{(1+t^2)^2}\right)\\
& = \left(\frac{4at}{(1+t^2)^2}, \frac{6at^2 + 2at^4}{(1+t^2)^2}\right).
\end{align*}
Since $(1+t^2)^2 > 0$ for all $t \in \RR$, we thus have $\alpha'(t) = 0$ if and only if $4at = 0$ and $6at^2 + 2at^4 = 0$. Then the only singular point of the cissoid is $\alpha(0) = (0, 0)$.

\item We have that \[\lim_{t\to\pm\infty}\frac{2at^2}{1+t^2} = 2a\] so $\alpha(t)$ approaches the line $x = 2a$ as $t\to\pm\infty$. Moreover, \[\lim_{t\to\pm\infty}\alpha'(t) = \left(\lim_{t\to\pm\infty}\frac{4at}{(1+t^2)^2}, \lim_{t\to\pm\infty}\frac{6at^2+2at^4}{(1+t^2)^2}\right) = (0, 2a).\]
\end{enumerate}
\end{ex}

\begin{ex}
\begin{enumerate}
\item We have that $\sin t, \cos t,$ and $\tan(t/2)$ are differentiable for all $t \in (0, \pi)$. Moreover, $\tan(t/2)$ is positive for $t \in (0, \pi)$, and $\log$ is differentiable on the positive reals. Thus $\alpha$ is a differentiable parameterized curve with
\begin{align*}
\alpha'(t) & = \left(\cos t, -\sin t + \frac{\frac{1}{2}\sec^2\frac{t}{2}}{\tan\frac{t}{2}}\right)\\
& = \left(\cos t, -\sin t + \frac{1}{2\sin\frac{t}{2}\cos\frac{t}{2}}\right)\\
& = \left(\cos t, -\sin t + \frac{1}{\sin t}\right)\\
& = (\cos t, \cos t\cot t)
\end{align*}
for all $t \in (0, \pi)$. Then $\alpha'(t) = 0$ if and only if $\cos t = 0$, that is, $t = \pi/2$. Hence $\alpha$ is regular except at $t = \pi/2$.

\item We have that for all $t \in (0, \pi)$ with $t\not = \pi/2$, the tangent line to $\alpha$ at $t$ is given by $\beta: \RR \to \RR^2$ where
\begin{align*}
\beta(s) & = \alpha(t) + s\alpha'(t)\\
& = \left(\sin t, \cos t + \log\tan\frac{t}{2}\right) + s(\cos t, \cos t\cot t)\\
& = \left(\sin t + s\cos t, (1 + s\cot t)\cos t + \log\tan\frac{t}{2}\right).
\end{align*}
Then the tangent line intersects the $y$-axis when \[\sin t + s\cos t = 0,\] that is, when $s = -\tan t$ (since $ t\not = \pi/2$, $\cos t \not = 0$). Then at time $t$, the length of the segment of the tangent line from the point of tangency to the $y$-axis is given by
\begin{align*}
\|\alpha(t) - \beta(-1)\| & = \left\|\left(\sin t, \cos t + \log\tan\frac{t}{2}\right) - \left(\sin t + s\cos t, (1 + s\cot t)\cos t + \log\tan\frac{t}{2}\right)\right\|\\
& = \|(\sin t, -s\cot t\cos t)\|\\
& = \|(\sin t, \cos t)\|\\
& = 1.
\end{align*}
\end{enumerate}
\end{ex}

\begin{ex}
\begin{enumerate}
\item We have that for $t \in (-1, \infty)$, $1+t^3 \not = 0$ and so
\begin{align*}
\alpha'(t) & = \left(\frac{(3a)(1+t^3) - (3at)(3t^2)}{(1+t^3)^2}, \frac{(6at)(1+t^3) - (3at^2)(3t^2)}{(1+t^3)^2}\right)\\
& = \left(\frac{3a - 6at^3}{(1+t^3)^2}, \frac{6at - 3at^4}{(1+t^3)^2}\right).
\end{align*}
In particular, \[\alpha'(0) = (3a, 0),\] and so at $t = 0$, $\alpha$ is tangent to the $x$-axis.

\item We have that \[\lim_{t\to\infty}\frac{3at}{1+t^3} = 0\] and \[\lim_{t\to\infty}\frac{3at^2}{1+t^3} = 0.\] Thus $\alpha(t) \to (0, 0)$ as $t\to\infty$. Moreover, $(1+t^3)^2 = 1 + 2t^3 + t^6$ is a degree 6 polynomial in $t$, and so \[\lim_{t\to\infty}\frac{3a-6at^3}{(1+t^3)^2} = 0\] and \[\lim_{t\to\infty}\frac{6at-3at^4}{(1+t^3)^2} = 0.\] Hence $\alpha'(t) \to (0, 0)$ as $t\to\infty$.

\item For $t \in (-1, \infty)$, we have
\begin{align*}
\alpha_1(t) + \alpha_2(t) + a & = \frac{3at}{1+t^3} + \frac{3at^2}{1+t^3} + a\\
& = \frac{3at(1+t)}{1+t^3} + a\\
& = \frac{3at}{1-t+t^2} + a\\
& = a\frac{1 + 2t + t^2}{1-t+t^2}.
\end{align*}
Hence \[\lim_{t\to -1^+}(\alpha_1(t) + \alpha_2(t) + a) = a\lim_{t\to -1^+}\frac{1+2t+t^2}{1-t+t^2} = 0.\] Morover, we computed in part a. that \[\alpha'(t) = \left(\frac{3a-6at^3}{(1+t^3)^2}, \frac{6at-3at^4}{(1+t^3)^2}\right).\] Then assuming that $a \not = 0$, we have \[\frac{\alpha_2'(t)}{\alpha_1'(t)} = \frac{6at-3at^4}{3a-6at^3} = \frac{6t-3t^4}{3-6t^3}\] and so \[\lim_{t\to-1^+}\frac{\alpha_2'(t)}{\alpha_1'(t)} = \frac{-6-3}{3+6} = -1.\] Thus as $t \to -1$, $\alpha$ and its tangent approach the line $x + y + a = 0$.
\end{enumerate}
\end{ex}

\begin{ex}
\begin{enumerate}
\item Since $a > 0$ and $b < 0$, $ae^{bt} \to 0$ as $t\to\infty$ with $ae^{bt} > 0$ for all $t \in \RR$. Then since $(\cos t, \sin t)$ traverses the unit circle counterclockwise, we have as $t\to\infty$ that $\alpha$ spirals counterclockwise around the origin with $\alpha(t) \to 0$ (since $\cos t, \sin t$ are bounded).

\item We compute for all $t \in \RR$ that \[\alpha'(t) = ae^{bt}(-\sin t + b\cos t, \cos t + b\sin t).\] Then from $ae^{bt} \to 0$ as $t\to\infty$ (with the remaining terms being bounded), we have that $\alpha'(t) \to (0, 0)$. Additionally, since $ae^{bt} > 0$ for all $t \in \RR$ we have
\begin{align*}
|\alpha'(t)| & = ae^{bt}|(-\sin t + b\cos t, \cos t + b\sin t)|\\
& = ae^{bt}\sqrt{(-\sin t + b\cos t)^2 + (\cos t + b\sin t)^2}\\
& = ae^{bt}\sqrt{1 + b^2}.
\end{align*}
Thus for any $t_0 \in \RR$, the arc length of $\alpha$ along $[t_0, \infty)$ is
\begin{align*}
\int_{t_0}^{\infty}|\alpha'(t)|\,dt & = \int_{t_0}^{\infty}ae^{bt}\sqrt{1+b^2}\,dt\\
& = a\sqrt{1+b^2}\int_{t_0}^{\infty}e^{bt}\,dt\\
& = a\sqrt{1+b^2}\lim_{t\to\infty}\frac{e^{bt} - e^{bt_0}}{b}\\
& = -\frac{a\sqrt{1+b^2}e^{bt_0}}{b}.
\end{align*}
In particular, the arc length of $\alpha$ along $[t_0, \infty)$ is finite.
\end{enumerate}
\end{ex}

\begin{ex}
\begin{enumerate}
\item For all nonzero $h \in \RR$, the line determined by $\alpha(0 + h) = (h^3, h^2)$ and $\alpha(0) = (0, 0)$ is given by $x = hy$. This line has limit position $x = 0$ as $h \to 0$, so $\alpha$ has a weak tangent at $t = 0$. But for nonzero $h \in \RR$, the line determined by $\alpha(0 + h) = (h^3, h^2)$ and $\alpha(0 - h) = (-h^3, h^2)$ is given by $y = h^2$ which has limit position $y = 0$. Hence $\alpha$ does not have a strong tangent at $t = 0$.

\item For all nonzero $h, k \in \RR$, by the mean value theorem there exists $t$ between $h$ and $k$ such that \[\frac{\alpha'(t_0 + h) - \alpha'(t_0 + k)}{h-k} = \alpha'(t).\] Hence by continuity of $\alpha'$, \[\lim_{h, k \to 0}\frac{\alpha'(t_0 + h) - \alpha'(t_0 + k)}{h-k} = \alpha'(t_0).\] Then since $\alpha'(t_0) \not = 0$, the line determined by $\alpha(t_0 + h)$ and $\alpha(t_0 + k)$ converges to the line through $\alpha(t_0)$ in the direction of $\alpha'(t_0)$ as $h, k \to 0$.

\item We have that \[\lim_{h\to 0^+}\frac{\alpha(0 + h)-\alpha(0)}{h} = \lim_{h\to 0^+}\frac{(h^2, h^2)-(0, 0)}{h} = \lim_{h\to 0^+}(h, h) = (0, 0)\] and \[\lim_{h\to 0^-}\frac{\alpha(0 + h)-\alpha(0)}{h} = \lim_{h\to 0^-}\frac{(h^2, -h^2) - (0, 0)}{h} = \lim_{h\to 0^-}(h, -h) = (0, 0)\] Then $\alpha'(0) = (0, 0)$ and so \[\alpha'(t) = \begin{cases}
(2t, 2t) & \text{if } t \geq 0\\
(2t, -2t) & \text{if } t \leq 0.
\end{cases}\] Clearly $\alpha'$ is continuous on $\RR_{\geq 0}$ and $\RR_{\leq 0}$ since \[\lim_{t\to 0^+}\alpha'(t) = \alpha'(0) = \lim_{t\to 0^-}\alpha'(t),\] and so $\alpha'$ is continuous on $\RR$. Thus $\alpha$ is $C^1$. But for $h > 0$, \[\frac{\alpha'(0 + h) - \alpha'(0)}{h} = \frac{(2h, 2h) - (0, 0)}{h} = (2, 2)\] while for $h < 0$, \[\frac{\alpha'(0 + h) - \alpha'(0)}{h} = \frac{(2h, -2h) - (0, 0)}{h} = (2, -2).\] Then $\alpha''(0)$ does not exist and hence $\alpha$ is not $C^2$. The trace of $\alpha$ is the graph of $x = |y|$.
\end{enumerate}
\end{ex}

\begin{ex}
Since $|\alpha'|$ is continuous on $[a, b]$, the integral $\int_a^b\|\alpha'(t)\|\,dt$ exists and so there exists $\delta > 0$ such that for $|P| < \delta$, \[\left|\int_a^b\|\alpha'(t)\|\,dt - \sum_{i = 1}^n(t_i-t_{i-1})\|\alpha'(t_{i-1})\|\right| < \epsilon/2.\] Since $\|\alpha'\|$ is continuous on the compact set $[a, b]$, it is uniformly continuous, and hence making $\delta > 0$ smaller if necessary yields \[|\|\alpha'(s)\|-\|\alpha'(t)\|| < \frac{\epsilon}{2(b-a)}\] for $s, t \in [a, b]$ with $|s-t| < \delta$. Then for $|P| < \delta$ we have \[\left|\|\alpha'(t_{i-1})\| - \left\|\frac{\alpha(t_i)-\alpha(t_{i-1})}{t_i-t_{i-1}}\right\|\right| < \frac{\epsilon}{2(b-a)}\] for all $i = 1, \ldots, n$ by the mean value theorem. Thus \[|(t_i-t_{i-1})\|\alpha'(t_{i-1})\| - \|\alpha(t_i)-\alpha(t_{i-1})\|| < (t_i-t_{i-1})\frac{\epsilon}{2(b-a)}\] and so \[\left|\sum_{i = 1}^n(t_i-t_{i-1})\|\alpha'(t_{i-1})\| - l(\alpha, P)\right| < \sum_{i = 1}^n(t_i-t_{i-1})\frac{\epsilon}{2(b-a)} = \epsilon/2.\] Hence \[\left|\int_a^b\|\alpha'(t)\|\,dt - l(\alpha, P)\right| < \epsilon/2 + \epsilon/2 = \epsilon,\] proving the claim.
\end{ex}

\begin{ex}
\begin{enumerate}
\item Following Exercise 8, we define the arc length of $\alpha$ along $[a, b] \subset I$ by \[\lim_{|P|\to\delta}l(\alpha, P)\] if it exists, where the limit is over partitions $P$ of $[a, b]$. Exercise 8 shows that if $\alpha$ is $C^1$, then this limit exists and \[\lim_{|P|\to\delta}l(\alpha, P) = \int_a^b\|\alpha'(t)\|\,dt.\]

\item Clearly $\alpha$ is differentiable on $(0, 1)$ and continuous at $t = 1$. At $t = 0$, we have \[\lim_{t\to 0^+}\alpha(t) = \left(\lim_{t\to 0^+}t, \lim_{t\to 0^+}t\sin\left(\frac{\pi}{t}\right)\right) = (0, 0) = \alpha(0)\] since $\sin$ is bounded. Thus $\alpha$ is $C^0$. We assume that a ``reasonable definition'' of arc length of a curve $\alpha$ of class $C^0$ on an interval $[a, b]$ has the property that for any partition $P$ of $[a, b]$, the arc length is at least $l(\alpha, P)$. For $n \in \NN$, consider the partition \[\frac{1}{n+1} < \frac{1}{n+\frac{1}{2}} < \frac{1}{n}\] of the interval $[1/(n+1), 1/n]$. Then the arc length of $\alpha$ along $[1/(n+1), 1/n]$ is at least
\begin{align*}
l(\alpha, P) & = \left\|\alpha\left(\frac{1}{n+\frac{1}{2}}\right) - \alpha\left(\frac{1}{n+1}\right)\right\| + \left\|\alpha\left(\frac{1}{n}\right) - \alpha\left(\frac{1}{n+\frac{1}{2}}\right)\right\|\\
& = \left\|\left(\frac{1}{n+\frac{1}{2}}, \frac{(-1)^n}{n+\frac{1}{2}}\right) - \left(\frac{1}{n + 1}, 0\right)\right\| + \left\|\left(\frac{1}{n}, 0\right) - \left(\frac{1}{n+\frac{1}{2}}, \frac{(-1)^n}{n+\frac{1}{2}}\right)\right\|\\
& \geq \left|\frac{(-1)^n}{n+\frac{1}{2}}\right| + \left|\frac{(-1)^n}{n+\frac{1}{2}}\right|\\
& = \frac{2}{n+\frac{1}{2}}.
\end{align*}
Hence the arc length of $\alpha$ on $[1/N, 1]$ for any $N \in \NN$ is at least \[\sum_{n = 1}^N\frac{2}{n+\frac{1}{2}} \geq 2\sum_{n = 1}^N\frac{1}{n+1}.\] But $\sum_{n = 1}^{\infty}\frac{1}{n+1}$ diverges, so the arc length of $\alpha$ on $[1/N, 1]$ tends to infinity as $N\to\infty$.
\end{enumerate}
\end{ex}

\begin{ex}
\begin{enumerate}
\item It is immediate from the bilinearity of the dot product that \[(q-p)\cdot v = \left(\int_a^b\alpha'(t)\,dt\right)\cdot v = \int_a^b\alpha'(t)\cdot v\,dt.\] Since $\|v\| = 1$, we have $\alpha'(t)\cdot v \leq \|\alpha'(t)\|$ for all $t \in I$ and hence \[\int_a^b\alpha'(t)\cdot v\,dt \leq \int_a^b\|\alpha'(t)\|\,dt.\] The right-hand side is the arc length of $\alpha$ on $[a, b]$.

\item From the inequality in part a., we find that \[(q-p)\cdot\frac{q-p}{\|q-p\|} \leq \int_a^b\|\alpha'(t)\|\,dt.\] The left-hand side is simply \[\frac{\|q-p\|^2}{\|q-p\|} = \|q-p\| = \|\alpha(b)-\alpha(a)\|,\] so \[\|\alpha(b)-\alpha(a)\| \leq \int_a^b\|\alpha'(t)\|\,dt.\]
\end{enumerate}
\end{ex}